\documentclass[12pt]{article}
\setlength{\oddsidemargin}{0in}
\setlength{\evensidemargin}{0in}
\setlength{\textwidth}{6.5in}
\setlength{\parindent}{0in}
\setlength{\parskip}{\baselineskip}

\usepackage{amsmath,amsfonts,amssymb,graphicx,xcolor,mathtools,mathabx,siunitx}
\sisetup{tight-spacing=true}

\newcommand{\purple}[1]{{\color{purple} #1}}

\begin{document}

PHYS 374 Fall 2020\hfill Worksheet 9: Planetary Wobbles\\
\\
Name:\\
\\
Please submit as a PDF on Moodle. Include any calculations made using external tools.

\hrulefill
\\
\\
Earth interacts with the Sun via a gravitational potential:
$$
U(r) = -\frac{G m_\Earth m_\Sun}{r}
$$

\begin{enumerate}
    \item Show that the equation of motion is as follows. Explain the significance of each term.
    $$
    \mu \ddot{r} = -\frac{d}{dr} \left[ \frac{\ell^2}{2 \mu r^2} -\frac{G \mu M}{r} \right]
    $$
    \item Sketch the gravitational potential, the centripetal potential, and the effective potential.
    \item Find the value $r_0$ at which $\ddot{r}=0$ and explain its significance. It's a bit messy, so be sure to double check your units!
    \item Approximate the effective potential near $r_0$ as a quadratic. Recall:
    $$
    f(x) \approx f(x_0) + f'(x_0) \; (x - x_0) + \tfrac{1}{2} f''(x) \; (x - x_0)^2 + \cdots
    \quad\quad\text{near $x_0$}
    $$
    \item Perform a change of variables $(r - r_0) \rightarrow \epsilon$. Show that the equation of motion can now be written as:
    $$
    \ddot{\epsilon} = - \omega^2 \epsilon
    $$
    \item Solve the equation of motion for $\epsilon(t)$. What is the value of $\omega$? How does this number line up with the motion of the Earth?
\end{enumerate}

The values below may be useful. Note $\ell_\Earth$ is Earth's orbital angular momentum.
\begin{align*}
    m_\Sun &= \SI{1.99e30}{\kilo\gram} &
    G &= \SI{6.67e-11}{\newton\meter^2/\kilogram^2} & \\
    m_\Earth &= \SI{5.97e24}{\kilo\gram} & 
    \ell_\Earth &= \SI{2.66e40}{\joule \second}
\end{align*}

\end{document}