\documentclass{article}
\usepackage[utf8]{inputenc}
\usepackage{amsmath,graphicx,hyperref,xcolor}

\setlength{\parindent}{0in}
\setlength{\parskip}{1em}

\usepackage{fancyhdr}
\rhead{}

\pagestyle{fancy}
\lhead{Physics 374B Final Exam}
\rhead{Fall 2020}
\cfoot{\thepage}

\begin{document}

Name: \makebox[2in]{\hrulefill}

\vspace{0.5in}

\begin{itemize}
    \item You should plan to spend about two hours on this exam. More time is allowed, but please don't spend all weekend on it.
    \item Turn in your work to the folder outside Darla's office (RNS 236) by noon Monday. If you're off campus, send me a \textbf{legible} scan of your work.
    \item Work on your own. You may not reference external resources such as your notes or the book.
    \item Number your pages. Put your name on each page.
    \item Each item is worth the same number of points. Some are more difficult than others.
    \item I have confidence in you!
\end{itemize}

\vfill

I pledge my honor that on this examination I have neither given nor received assistance not explicitly approved by the professor and that I have seen no dishonest work \makebox[2in]{\hrulefill}

\vspace{0.5in}

I have intentionally not signed this pledge \makebox[2in]{\hrulefill}

\newpage

\section*{Problem 1}

We are interested in the power output ($P$) of a wind turbine as a function of the mass density of air ($\rho$), the area swept out by the blades ($A$), and the velocity of the wind through the blades ($v$).

\begin{itemize}
    \item What are the units of power, mass density, area, and velocity? You may look these up if necessary.
    \item Suppose $P = \rho^{\alpha} A^{\beta} v^{\gamma}$, then write down a system of linear equations relating $\alpha$, $\beta$, and $\gamma$. 
    \item Solve the system of equations to write $P$ in terms of $\rho$, $A$, and/or $v$.
    \item The typical wind speed at St Olaf is 1\% faster than at Carleton. However, due to the increased altitude, the air density at St Olaf is also 1\% less dense\footnote{This is an exaggeration!}. Where would you expect a turbine to produce more power?
\end{itemize}

\newpage

\section*{Problem 2}

These questions refer to the video projects made by your classmates. 

\begin{itemize}
    \item Sketch a double pendulum and an asymmetric double torsion pendulum. Label any quantity that you would expect to show up in the Lagrangian. The letter $K$ is used for torsion constants (like spring constants for twisting).
    \item For a simple pendulum, a small-angle approximation is required to get a pencil-and-paper solution. Do you expect that would also be the case for a double pendulum? How about for an ADTP? Explain your reasoning.
    
    Hint: recall that no such approximation is necessary for the coupled oscillator.
    \item Argue briefly that the three-body problem has been solved.
    \item Argue briefly that the three-body problem has not been solved. 
    
\end{itemize}

\newpage

\section*{Problem 3}

A thin circular tire of mass $M$ and radius $R$ rolls without slipping on a flat surface. It rolls over a nail of mass $m$, which gets stuck in the tire. Let $X$ be the horizontal position of the center of the tire, and $\phi$ be the position of the nail on the wheel, measured clockwise from the bottom.

\begin{itemize}
    \item Draw a picture! Make sure $X$ and $\phi$ are clearly marked.
    \item Argue (using words) that $X = R\phi$. Is this a suitable constraint for using a Lagrange multiplier? If so, what force of constraint would the multiplier $\lambda$ correspond to?
    \item Write down the kinetic energy (translation and rotation) of the tire. Show that it can be rearranged to look like:
    \begin{align*}
        T_\text{tire} = M R^2 \dot{\phi}^2
    \end{align*}
    \item What is the position $(x, y)$ of the nail in terms of $\phi$? Use that to calculate its velocity, $(\dot{x}, \dot{y})$, and show:
    \begin{align*}
        T_\text{nail} = m R^2 \dot{\phi}^2 \left( 1 - \cos \phi \right) 
    \end{align*}
    \item Write down the Lagrangian, and show that the equation of motion is:
    \begin{align*}
        m R^2 \dot{\phi}^2 \sin\phi - m g R \sin\phi = 2 (M+m) R^2 \ddot{\phi} - 2 m R^2 \ddot{\phi} \cos\phi + 2 m R^2 \dot{\phi}^2 \sin\phi
    \end{align*}
    \item Approximate $\sin\phi$ and $\cos\phi$ for the case where $\phi \approx 0$, and neglect any term that is quadratic or higher in $\phi$, $\dot{\phi}$ and/or $\ddot{\phi}$. Show the result is:
    \begin{align*}
        \ddot{\phi} = -\frac{m g}{2 M R} \phi
    \end{align*}
    \item Solve the above equation for $\phi(t)$ and describe the motion in words. Comment on limiting cases for $g$, $R$, and the ratio of the masses. Do these make sense?
    \item Now go back to the full equation of motion and approximate $\sin\phi$ and/or $\cos\phi$ when $\phi \approx \pi$. Show the result is:
    \begin{align*}
        \ddot{\varepsilon} &= \frac{mg}{2R(M+2m)} \varepsilon & &\text{where $\varepsilon=\phi - \pi$}
    \end{align*}
    \item Solve the equation above for $\varepsilon(t)$ and describe the motion in words. Comment on limiting cases. 
    
\end{itemize}

\newpage

\section*{Problem 4}

Two bodies interact via a central potential of the form:
\begin{align*}
    U(r) &= \alpha r^7
\end{align*}

\begin{itemize}
    \item The equation of motion for two bodies interacting via a central potential $U(r)$ can be found on the last page. Briefly explain the significance of the constants $\ell$ and $\mu$. Also explain how it's possible for us to explain the motion of two bodies using only a single coordinate, $r$. 
    \item Sketch the potential $U(r)$, the centripetal ``potential,'' and the effective potential. Based on that sketch, do you expect a circular orbit to be stable or unstable? Explain your reasoning.
    \item Show that the radius of a circular orbit is given by:
    \begin{align*}
        r_0^9 &= \frac{\ell^2}{7 \alpha \mu}
    \end{align*}
    \item Show that the effective potential near $r_0$ can be approximated as:
    \begin{align*}
        U_\text{eff}(r) \approx \frac{9 \ell^2}{14 \mu r_0^2} + \frac{1}{2} \frac{9 \ell^2}{\mu r_0^4} \, (r - r_0)^2
    \end{align*}
    \item Using the above approximation, show that small perturbations from a circular orbit are described by:
    \begin{align*}
        \ddot{\varepsilon} &= - \frac{9 \ell^2}{\mu^2 r_0^4} \varepsilon &
        & \text{where $\varepsilon = r - r_0$}
    \end{align*}
    \item Show that the angular frequency of a circular orbit is given by:
    \begin{align*}
        \dot{\phi} = \frac{\ell}{\mu r_0^2}
    \end{align*}
    \item Sketch the path of an orbit perturbed slightly from $r_0$. Does the orbit close?
\end{itemize}

\newpage

\section*{For Reference}

When $x \approx x_0$, the function $f(x)$ can be approximated:
\begin{align*}
    f(x) = f(x_0) + f'(x_0) \, (x - x_0) + \tfrac{1}{2} f''(x_0) \, (x - x_0)^2 + \cdots
\end{align*}

Equation of motion for two bodies interacting via a central potential:
\begin{align*}
    \mu \ddot{r} = -\frac{d}{dr} \left[ \frac{\ell^2}{2 \mu r^2} + U(r) \right]
\end{align*}

Euler-Lagrange equation, true for each coordinate $x$ that $\mathcal{L}$ depends on:
\begin{align*}
    \frac{\partial \mathcal{L}}{\partial x} &= \frac{d}{dt} \frac{\partial \mathcal{L}}{\partial \dot{x}}
\end{align*}

Moment of inertia for a thin ring: $I = M R^2$

Orbital angular momentum: $\vec{\ell} = \vec{r} \times \vec{p}$, or alternatively $\ell = m r^2 \dot{\phi}$

\end{document}
