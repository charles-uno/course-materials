\documentclass{article}
\usepackage[utf8]{inputenc}
\usepackage{amsmath,amsfonts,amssymb,graphicx,xcolor,mathtools,framed}
\usepackage[noabbrev]{cleveref}

\setlength{\parindent}{0in}
\setlength{\parskip}{1em}

\usepackage{fancyhdr}
\rhead{}

\pagestyle{fancy}
\lhead{Physics 374B Notes}
\rhead{Fall 2020}
\cfoot{\thepage}

\renewcommand{\thefootnote}{\fnsymbol{footnote}}

% Add a background color to block quotes for emphasis
\let\oldquote=\quote
\let\endoldquote=\endquote
\colorlet{shadecolor}{orange!12}
\renewenvironment{quote}{\begin{shaded*}\begin{oldquote}}{\end{oldquote}\end{shaded*}}

\begin{document}

\section*{Lagrange Multipliers}

Let's say we're looking to maximize a quantity $U$ given by:
\begin{align}
    \label{defu}
    U = \displaystyle\int_{x_1}^{x_2} f(y, y', x) \; dx    
\end{align}
Note that primes show differentiation with respect to $x$. For example, $y' = \tfrac{dy}{dx}$.

Furthermore, say our system is constrained by an expression of the form:
\begin{align}
    \label{defc}
    C = \displaystyle\int_{x_1}^{x_2} g(y, y', x) \; dx
\end{align}
Where $C$ is some fixed constant.

Suppose $y(x)$ is our solution: a path that gives a stationary point in $U$ subject to our constraint $C$. Then consider an adjacent path:
\begin{align}
    \label{defy}
    Y(x) = y(x) + \alpha \, \eta(x)
\end{align}
We insist that $\eta(x_1) = \eta(x_2) = 0$. That is, $Y(x)$ and $y(x)$ must share the same endpoints. The function $\eta$ must also respect our constraint from \cref{defc}. Regardless of $\alpha$, $C$ must remain fixed:
\begin{align}
    \frac{dC}{d\alpha} = \frac{d}{d\alpha} \displaystyle\int_{x_1}^{x_2} g(y, y', x) \; dx = 0
\end{align}
Expanding via the chain rule and evaluating a few well-behaved terms:
\begin{align}
    \frac{dC}{d\alpha} &= \displaystyle\int_{x_1}^{x_2} \underbrace{ \frac{\partial g}{\partial \alpha} }_{0} + \frac{\partial g}{\partial y} \underbrace{ \frac{dy}{d\alpha} }_{\eta} + \frac{\partial g}{\partial y'} \underbrace{ \frac{dy'}{d\alpha} }_{\eta'} \; dx \notag \\
    &= \displaystyle\int_{x_1}^{x_2} \eta \frac{\partial g}{\partial y} + \eta' \frac{\partial g}{\partial y'} \; dx
\end{align}
Recall that $\eta=0$ at the endpoints, so we can use integration by parts to rearrange the second term, then factor out an $\eta$:
\begin{align}
\label{dcdalpha}
\frac{dC}{d\alpha}
=
\displaystyle\int_{x_1}^{x_2} \left[ \frac{\partial g}{\partial y} - \frac{d}{dx} \frac{\partial g}{\partial y'} \right] \eta \; dx    
\end{align}
Using identical steps, we can come up with an analogous expression for $\tfrac{dU}{d\alpha}$:
\begin{align}
\label{dudalpha}
\frac{dU}{d\alpha}
=
\displaystyle\int_{x_1}^{x_2} \left[ \frac{\partial f}{\partial y} - \frac{d}{dx} \frac{\partial f}{\partial y'} \right] \eta \; dx
\end{align}
If $\eta(x)$ we an arbitrary function, we could make the argument\footnote{
Let $\int \eta(x) h(x) dx = 0$, where $\eta$ is an arbitrary function. Then suppose $h(x_0) \not = 0$ in our domain. We can then construct $\eta(x)$ to be nonzero at $x_0$ but zero everywhere else. Hence the integral is nonzero, a contradiction. So $h(x)$ must be uniformly zero.
} that he bracketed quantities in \cref{dudalpha} must be uniformly zero, which would give the usual Euler-Lagrange equation. However, because of our constraint from \cref{defc}, $\eta(x)$ is not quite arbitrary and the bracketed expression may be nonzero.

We resolve this issue by combining our two expressions, with the aid of an undetermined multiplier $\lambda$:
\begin{align}
    \frac{dU^*}{d\alpha} &= \frac{dU}{d\alpha} + \lambda \frac{dC}{d\alpha} \notag \\
    \label{defustar}
    &= \displaystyle\int_{x_1}^{x_2} \left[ \frac{\partial f}{\partial y} - \frac{d}{dx} \frac{\partial f}{\partial y'} \right] \eta \; dx + \lambda \displaystyle\int_{x_1}^{x_2} \left[ \frac{\partial g}{\partial y} - \frac{d}{dx} \frac{\partial g}{\partial y'} \right] \eta \; dx \notag \\
    &= \displaystyle\int_{x_1}^{x_2} \left[ \frac{\partial f}{\partial y} - \frac{d}{dx} \frac{\partial f}{\partial y'} + \lambda \frac{\partial g}{\partial y} - \lambda \frac{d}{dx} \frac{\partial g}{\partial y'} \right] \eta \; dx \notag \\
    &= \displaystyle\int_{x_1}^{x_2} \left[ \frac{\partial (f + \lambda g)}{\partial y} - \frac{d}{dx} \frac{\partial (f + \lambda g)}{\partial y'} \right] \eta \; dx
\end{align}
Recall that $\tfrac{dU}{d\alpha} = 0$ at the stationary point $\alpha=0$, and $\tfrac{dC}{d\alpha}=0$ as long as $\eta$ is consistent with the constraint from \cref{defc}. Therefore their sum must be zero at the stationary point as long as $\eta$ respects the constraint. Therefore:
\begin{align}
    \displaystyle\int_{x_1}^{x_2} \left[ \frac{\partial (f + \lambda g)}{\partial y} - \frac{d}{dx} \frac{\partial (f + \lambda g)}{\partial y'} \right] \eta \; dx = 0 \quad\quad\text{at $\alpha=0$}
\end{align}
\begin{quote}
    From here, we have to argue that the bracketed quantity is uniformly zero. We know this is true, but the logic isn't quite clicking. The multiplier $\lambda$ isn't much help since it's constant, rather than a function of $x$. By including $g$, is the constraint ``baked in" to $\eta$ so we can again treat it as an arbitrary function? 
\end{quote}
We can further note that this is the Euler-Lagrange equation for an integrand $f + \lambda g$. So we may equivalently write that a stationary point in $U$ from \cref{defu}, respecting the constraint from \cref{defc}, is a stationary point of $U^*$:
\begin{align}
    U^* = \displaystyle\int_{x_1}^{x_2} f(y, y', x) + \lambda \, g(y, y', x) \; dx
\end{align}

\end{document}