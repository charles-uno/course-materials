\documentclass{article} 

\usepackage[percent]{overpic}

\usepackage{graphicx}
\usepackage{xcolor}
\definecolor{AccentColor}{RGB}{124, 98, 177}
\usepackage{hyperref}
\hypersetup{
    colorlinks=true,
    % internal links
    linkcolor=AccentColor,
    filecolor=AccentColor,      
    % external links
    urlcolor=AccentColor,
    }

\setlength\parindent{0pt}
\usepackage{parskip} 

\title{CS 241: Hardware Design}
\author{Professor Charles Fyfe}
\date{St Olaf College $\cdot$ Spring 2026}

\begin{document}

\maketitle
\tableofcontents

\newpage

\begin{overpic}[width=\textwidth]{dune-robot-hbo}
	\put (4,8) {\textcolor{white}{\textbf{Thou shalt not make a machine in the likeness of a human mind}}}
	\put (70, 3) {\textcolor{white}{Dune, Frank Herbert}}
\end{overpic}


\section{Specific to This Course}

\subsection{Course Description}

Computers sometimes feel like magic! You write code (or maybe even a sentence)
and it figures out complex tasks. But it's not magic. It's not alive. It's just
a few layers of physics and logic.

This course is about breaking down the fundamental components and processes
that power computers. In particular:
\begin{itemize}
	\item The basic components of a computer
	\item How data is stored using ones and zeroes
	\item Building electrical circuits to perform math and logic
	\item Coding exercises in Assembly (a very low-level language)
\end{itemize}


\subsection{Course Schedule}

See Moodle

\subsection{Office Hours}

TBD. Probably an hour right after class.

\subsection{Grading}

\begin{itemize}
	\item 45\% Tests
	\item 45\% Assignments
	\item 10\% Peer Reviews
\end{itemize}

\textbf{Tests.} This class has four tests, each covering two standards. All eight standards are covered again on the final. Each standard is graded as follows:
\begin{itemize}
	\item You have demonstrated proficiency. Full credit. Well done!
	\item You are close! Half credit. Revise within a week for full credit
	\item Partial proficiency. Half credit. Please try again on the final
	\item You have not yet demonstrated proficiency. Please try again on the final
\end{itemize}

You only have to demonstrate proficiency once per standard! If you get it on the test, you get to skip that part of the final. If you demonstrate proficiency on the final, you get full credit for that standard regardless of what your score was on the test.

Tests are take-home. No book, no notes, no computer, no nothing. Just paper and pencil. See the calendar on Moodle for dates.

\textbf{Assignments.} There is one assignment per standard. Assignments are found on Doenet. There will be a few different types of work. A single assignment will often include a mix of the following:
\begin{itemize}
	\item Worksheets graded automatically via Doenet
	\item Work submitted as a PDF via Moodle. This may include pencil-and-paper work, screenshots of circuit simulations, etc
	\item Code exercises submitted via CSGit. We will set this up in class
\end{itemize}

The prompt on Doenet should make clear how each part of the assignment is to be submitted. Please contact me if you encounter ambiguity!

\textbf{Peer reviews.} At the end of the course, you will each write reviews for a few of your peers. These reviews will be short. Less than one page. You will provide \emph{specific examples} of ways your peers helped you succeed.

Show up. Work together. Make it easy for your peers to write nice things about you. I recommend also keeping some light notes over the course of the semester. Write down whenever one of your peers is particularly helpful or insightful.

\subsection{Final Exam}

May 15th at 9am

\subsection{Late Work Policy}

Please try to get your work in on time. Otherwise you are likely to fall behind in the course.

Late work is inconvenient for the graders. They may be grumpy and vindictive when grading late work.

No late work will be accepted for the last two weeks of the semester. The graders have their own finals to worry about!

When submitting late work, you must notify the graders by email. Otherwise they will not know to look for it.

\subsection{Important Links}

\begin{itemize}
	\item Dive Into Systems: \url{https://diveintosystems.org/book/index.html}
	\item Doenet: \url{https://www.doenet.org/course?tool=dashboard&courseId=_GnqAk2zB64CHKPeZY9Ren}
	\item ARM Tutorial: \url{https://diveintosystems.org/book/C9-ARM64/index.html}
	\item ARM Simulator: \url{http://163.238.35.161/~zhangs/arm64simulator/}
	\item Circuitverse: \url{https://circuitverse.org/simulator}
\end{itemize}


\section{Bigger Picture}


\subsection{Academic Integrity}

Plagiarism is a serious academic offense. Hand in your own work. Give credit appropriately when you draw from the work of others. For more information please see:

\begin{itemize}
	\item St Olaf Honor Code: \url{https://wp.stolaf.edu/honorcouncil/}
	\item Faculty Handbook on Academic Integrity: \url{https://wp.stolaf.edu/facultyhandbook/academic-integrity-faculty-handbook-category-2}
	\item Roadmap to Academic Integrity: \url{https://wp.stolaf.edu/roadmap-to-academic-integrity}
\end{itemize}

Work that violates this policy will typically receive no credit. In especially serious cases the penalty can be an F in the course.

\subsection{AI Usage Policy}

You are welcome to use AI tools when studying.

Please not submit AI work as your own.

\subsection{Disability Accommodation}

I am committed to supporting the learning of all students in my class. If you have already
registered with Disability and Access (DAC) and have your letter of accommodations, please
meet with me as soon as possible to discuss, plan, and implement your accommodations in the
course. If you have or think you have a disability (learning, sensory, physical, chronic health,
mental health or attentional), please contact Disability and Access staff at 507-786-3288 or by visiting the DAC website: \url{https://wp.stolaf.edu/academic-support/dac}.

If you have an accommodation that allows you extra time or a low distraction environment for
quizzes and exams, please email me at least three days before each quiz or exam, so that I can
make sure to reserve a room.

\subsection{Gender Pronouns}

This course affirms people of all gender expressions and gender identities. If you go by a
different name than what is on the class roster, please let me know.

Using correct gender
pronouns is important to me. You are encouraged to share your pronouns with me and
correct me if I make a mistake. If you have any questions or concerns, please do not hesitate
to contact me.

\subsection{Land Acknowledgement}

We stand on the homelands of the Wahpekute Band of the Dakota Nation. We honor with
gratitude the people who have stewarded the land throughout the generations and their ongoing
contributions to this region. We acknowledge the ongoing injustices that we have committed
against the Dakota Nation, and we wish to interrupt this legacy, beginning with acts of healing
and honest storytelling about this place.

For more information about land acknowledgement statements at St Olaf, please see \url{https://wp.stolaf.edu/education/land-acknowledgement/}{here} and \href{https://wp.stolaf.edu/equity-inclusion/land-acknowledgement/}.

\subsection{Mental Health}

I greatly value your experience in this class, and it is my duty to facilitate a safe, caring, and
productive learning environment.

I recognize that you may experience a range of emotional, physical, and/or psychological
issues, both in and out of the classroom, that may distract you from your learning.

If you are experiencing such issues, please do not hesitate to come see me. I am here to listen.
We can also discuss what further resources might be available to you.

\subsection{Multilingual Support}

I am committed to making course content accessible to all students. If English is not your first
language and this causes you concern about the course, please speak with me. Students who
would like extra support with writing or speaking in English can also contact the language support specialist (\href{mailto:berryag@stolaf.edu}{berryag@stolaf.edu}) in the Academic Success Center.

\subsection{Religious Accommodation}

As part of my commitment to make St Olaf an inclusive community, I will provide students with
reasonable religious accommodations. If you will be missing class for a religious observance or
require another religious accommodation, please meet with me to discuss these.

\subsection{Required Referrals}

You are welcome to talk to me about circumstances outside the course that
affect your classroom experience or acadmic performance. However, please keep
in mind that I am required to refer cases of discrimination, harassment,
sexual misconduct, and violence.

Here are some resources where you can share privately:

\begin{itemize}
	\item Boe House Counseling Center: \url{https://wp.stolaf.edu/counseling-center/}
	\item College Pastors and Chaplains: \url{https://wp.stolaf.edu/ministry/}
	\item Health Services: \url{https://wp.stolaf.edu/health/}
	\item Sexual Assault Resource Network: \url{https://pages.stolaf.edu/sarn/}
	\item TimelyCare: \url{https://wp.stolaf.edu/timelycare/}
\end{itemize}

\subsection{Statement of Inclusivity}

In keeping with St Olaf College's mission statement, this class strives to be an inclusive
learning community, respecting those of differing backgrounds and beliefs. As a community, we
aim to be respectful to all citizens in this class, regardless of race, ethnicity, religion, gender or
sexual orientation.

\subsection{St Olaf Pride Statement}

As an Ole, I will practice:
\begin{itemize}
	\item Passion for learning and pursuit of vocation
	\item Respect for the worth and dignity of all people
	\item Integrity at all times, in all circumstances
	\item Dedication to a life of service, and
	\item Engagement with my community and the world.
\end{itemize}


\end{document}