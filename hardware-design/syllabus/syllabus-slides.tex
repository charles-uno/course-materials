
\Section{Syllabus}

\begin{frame}{Course Overview}

	\includegraphics[width=\columnwidth]{images/dune-robot-hbo}

	{\center
		Thou shalt not make a machine in the likeless of a human mind
	}

	{\small \hfill
		The Orange Catholic Bible, Dune, Frank Herbert
	}
\end{frame}


\begin{frame}{Course Overview}

	Computers feel like magic. You type words. The machine does complex things. Especially in the era of large language models

	But it's not magic. Computers are made of semiconductors, metal, and magnets. Three different kinds of rocks

	This course is about demystifying the machine and exploring how, fundamentally, computers do what they do. Specifically, we'll be looking at:
	\begin{itemize}
		\item The fundamental components of a computer
		\item How data is stored on a computer
		\item Building electrical circuits to perform logic
		\item Programming in Assembly (a very low-level language)
	\end{itemize}
\end{frame}

\Subsection{Grading}

\begin{frame}{Course Grade Breakdown}
	\begin{itemize}
		\item Quizzes and final: 45\%
		\item Homework and labs: 45\%
		\item Participation: 10\%
	\end{itemize}
\end{frame}

\begin{frame}{Standards-Based Grading}
	\begin{itemize}
		\item There are nine standards in the class. Each is worth 5\% of your grade.
		\item There are three quizzes. Each quiz covers three standards.
		\item The final covers all nine standards.
		\item If you demonstrate proficiency on the quiz, you're done with that standard. Full credit. You can skip that part of the final.
		\item If you demonstrate partial proficiency on the quiz, you get half credit. Try again on the final for full credit.
		\item If you do not demonstrate proficiency on the quiz, you get no credit. You can still get full credit on the final!
	\end{itemize}
\end{frame}


\begin{frame}{Homework and Labs}
	\begin{itemize}
		\item One or more lab exercises for each standard. We start these together in class. You may need to finish on your own time.
		\item We will have a homework assignment for each standard.
		\item Please work together in small groups.
		\item Please ensure that the work you turn in reflects your own understanding.
	\end{itemize}
\end{frame}









