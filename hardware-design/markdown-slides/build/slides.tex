\documentclass{beamer}

\title{CS 241: Hardware Design}
\date{\today}
\author{Charles Fyfe}
\institute{St Olaf College}

\usetheme{grape}

\begin{document}

\titlepage

\TOC{}



\Section{Introduction}

\begin{frame}{Course Overview}


\includegraphics{images/dune-robot-hbo}


Thou shalt not make a machine in the likeness of a human mind
\begin{itemize}
\item 
Dune, Frank Herbert

\end{itemize}
\end{frame}



\Subsection{Grading}

\begin{frame}{Peer Reviews}
\begin{itemize}
\item 
Participation score (10\% of your total grade) is mostly based on peer reviews

\item 
Reviews will be short. Less than one page. They should include \textit{specific examples} of ways your peers helped you succeed

\item 
You can get full credit here without too much trouble. Show up. Work together. Make it easy for your peers to write nice things about you

\item 
I recommend keeping notes over the course of the semester when someone is particularly helpful, insightful, etc

\end{itemize}
\end{frame}


\begin{frame}{Important Links}
\begin{itemize}
\item 
Dive Into Systems: https://diveintosystems.org/book/index.html

\item 
Doenet: https://www.doenet.org/course?tool=dashboard\&courseId=\_GnqAk2zB64CHKPeZY9Ren

\item 
ARM Tutorial: https://diveintosystems.org/book/C9-ARM64/index.html

\item 
ARM Simulator: http://163.238.35.161/~zhangs/arm64simulator/

\item 
Circuitverse: https://circuitverse.org/simulator

\end{itemize}
\end{frame}



\Section{Data Representation}

\begin{frame}{How do computers store information?}

Computers work using electrical signals which can be on or off. If we let:
\begin{itemize}
\item 
On means 1

\item 
Off means 0

\end{itemize}

We can use sequences of 0s and 1s to represent information
\end{frame}
\end{document}