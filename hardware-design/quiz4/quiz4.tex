\documentclass[12pt]{article}
\usepackage[utf8]{inputenc}
\usepackage{alltt, amssymb, amsmath,graphicx,hyperref,xcolor}

\setlength{\parindent}{0in}
\setlength{\parskip}{1em}

\usepackage{fancyhdr}
\rhead{}

\pagestyle{fancy}
\lhead{St Olaf College}
\chead{CS 241}
\rhead{Spring 2024}

\cfoot{\thepage}

\begin{document}

Name: \makebox[3in]{\hrulefill}

\vfill

\begin{center}
{\huge Quiz 4}
\end{center}

\begin{itemize}
    \item The standards AP2 and AP3 do not cover global variables or terminal input/output. You will \underline{not} be penalized for syntax errors on those.
    \item Please make sure your work is legible. There is a blank page following each exercise. More paper is available if needed.
    \item I have confidence in you!
\end{itemize}

\vfill

I pledge my honor that on this examination I have neither given nor received assistance not explicitly approved by the professor and that I have seen no dishonest work 

\hfill Signed: \makebox[3in]{\hrulefill}

$\square$\quad I have intentionally not signed the pledge. (check only if appropriate)
\newpage

\section*{Standard: Assembly Programming II}

On the next page is a commented skeleton for an Assembly program. 

This program reads in two numbers from the terminal. It passes the \textit{memory addresses} of those values into the function \texttt{add\_two\_numbers\_by\_address}, along with a third memory address where the result should be stored. The function performs the addition, stores the result at the given address, and returns that address. Back in \texttt{main}, the program loads the result and prints it. It should look like this:

\begin{alltt}
    > ./add_by_address
    Enter a number: 42
    Enter another number: 51
    The sum of those numbers is: 93
\end{alltt}

Please fill in the missing code. 

A memory diagram is not required, but you may find it helpful.

\vfill

\rule[1ex]{\textwidth}{.1pt}

$\square$ \textbf{P}: You have demonstrated proficiency. Full credit. Well done!

$\square$ \textbf{R}: You are close! Half credit. Submit a revision for full credit

$\square$ \textbf{S}: Partial proficiency. Half credit. Try again on the exam

$\square$ \textbf{I}: You have not yet demonstrated proficiency. Try again on the exam

\newpage

\begin{alltt}
@ ascii global variables

@ define function add_two_numbers_by_address

    @ stack frame setup

    @ read the two input values from memory

    @ add them
    
    @ write the result to memory

    @ return the address of the result

    @ stack frame teardown

@ define function main

    @ stack frame setup

    @ print prompts, read values to memory

    @ set input values for the function call

    @ call add_two_numbers_by_address

    @ load the result from memory

    @ print the output

    @ return 0, stack frame teardown

@ pointers to ascii global variables
\end{alltt}

\newpage

\begin{center}
(blank page)
\end{center}

\newpage

\section*{Standard: Assembly Programming III}

On the following page is an Assembly program for a number guessing game. The program picks a ``random'' number between 0 and 127. The user has ten chances to guess it. After each wrong guess, the program tells if it was too high or too low. It looks like this:

\begin{alltt}
    > ./guessing_game
    I'm thinking of a number 0 to 127
    Guess #1: 23
    Too low!
    Guess #2: 78
    Too low!
    Guess #3: 105
    Too high!
    Guess #4: 87
    Too low!
    Guess #5: 93
    Too low!
    Guess #6: 95
    Correct!
\end{alltt}

Please fill in the missing logic. 

There are several correct ways to write the solution. You do not need to follow the comments exactly.

\vfill

\rule[1ex]{\textwidth}{.1pt}

$\square$ \textbf{P}: You have demonstrated proficiency. Full credit. Well done!

$\square$ \textbf{R}: You are close! Half credit. Submit a revision for full credit

$\square$ \textbf{S}: Partial proficiency. Half credit. Try again on the exam

$\square$ \textbf{I}: You have not yet demonstrated proficiency. Try again on the exam

\newpage

\begin{alltt}
    .section .rodata
intro: .ascii "I'm thinking of a number 0 to 127{\textbackslash}n{\textbackslash}0"
prompt: .ascii "Guess #%d: {\textbackslash}0"
input_format: .ascii "%d{\textbackslash}0"
too_high_reply: .ascii "Too high!{\textbackslash}n{\textbackslash}0"
too_low_reply: .ascii "Too low!{\textbackslash}n{\textbackslash}0"
correct_reply: .ascii "Correct!{\textbackslash}n{\textbackslash}0"
no_more_tries: .ascii "No more tries. Better luck next time!{\textbackslash}n{\textbackslash}0"

    .text
    .global main
main: 
    @ stack frame setup
    push \{fp, lr\}
    add fp, sp, #4
    sub sp, sp, #4
    @ use current time to generate "random" number in r4
    sub r0, fp, #8
    bl time
    ldr r4, [fp, #-8]
    and r4, r4, #0x7f
    @ print game intro
    ldr r0, intro_ptr
    bl printf
    @ init guess counter


    @ top of loop


    @ increment guess counter


    @ break if there are no more guesses


    @ print the numbered prompt


    @ read in the guess


    @ check guess, jump to appropriate response


    @ case: guess is too small


    @ case: guess is too large


    @ case: guess is correct


    @ case: no more guesses


    @ stack frame teardown, return 0
    add sp, sp, #4
    pop \{fp, lr\}
    mov r0, #0
    bx lr

intro_ptr: .word intro
prompt_ptr: .word prompt
input_format_ptr: .word input_format
too_high_reply_ptr: .word too_high_reply
too_low_reply_ptr: .word too_low_reply
correct_reply_ptr: .word correct_reply
no_more_tries_ptr: .word no_more_tries
\end{alltt}


\newpage

\begin{center}
(blank page)
\end{center}


\end{document}
