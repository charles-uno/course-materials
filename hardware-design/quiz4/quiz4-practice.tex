\documentclass[12pt]{article}
\usepackage[utf8]{inputenc}
\usepackage{alltt, amssymb, amsmath,graphicx,hyperref,xcolor}

\setlength{\parindent}{0in}
\setlength{\parskip}{1em}

\usepackage{fancyhdr}
\rhead{}

\pagestyle{fancy}
\lhead{St Olaf College}
\chead{CS 241}
\rhead{Spring 2024}

\cfoot{\thepage}

\begin{document}

Name: \makebox[3in]{\hrulefill}

\vfill

\begin{center}
{\huge Quiz 4 Practice}
\end{center}

\begin{itemize}
    \item This is a practice quiz. You do not need to sign the pledge. You do not need to turn it in. 
    \item I encourage you to try it alone first to test your own understanding. You are welcome to review with your peers after.
    \item This is probably a bit harder than the real quiz.
\end{itemize}

\vfill

I pledge my honor that on this examination I have neither given nor received assistance not explicitly approved by the professor and that I have seen no dishonest work 

\hfill Signed: \makebox[3in]{\hrulefill}

$\square$\quad I have intentionally not signed the pledge. (check only if appropriate)
\newpage

\section*{Standard: Assembly Programming II}

On the next page is the skeleton of an Assembly program with comments. 

This program reads in two numbers from the terminal. It passes the \textit{memory addresses} of those values into the function \texttt{add\_two\_numbers\_by\_address}, along with a third memory address where the result should be stored. The function performs the addition, stores the result at the given address, and returns that address. Back in \texttt{main}, the program loads the result and prints it. It should look like this:

\begin{alltt}
    > ./add_by_address
    Enter a number: 42
    Enter another number: 51
    The sum of those numbers is: 93
\end{alltt}

Please fill in the missing code. Use a blank sheet.

Memory diagrams are not required but you may find one helpful.

\vfill

\rule[1ex]{\textwidth}{.1pt}

$\square$ \textbf{P}: You have demonstrated proficiency. Full credit. Well done!

$\square$ \textbf{R}: You are close! Half credit. Submit a revision for full credit

$\square$ \textbf{S}: Partial proficiency. Half credit. Try again on the exam

$\square$ \textbf{I}: You have not yet demonstrated proficiency. Try again on the exam

\newpage

\begin{alltt}
@ ascii global variables

    .text
    .global add_two_numbers_by_address
add_two_numbers_by_address:
    @ stack frame setup

    @ read the two input values from memory

    @ add them
    
    @ write the result to memory

    @ return the address of the result

    @ stack frame teardown

    .text
    .global main
main: 
    @ stack frame setup

    @ print prompts, read values

    @ set input values for the function call

    @ call add_two_numbers_by_address

    @ load the result from memory

    @ print the output

    @ return 0, stack frame teardown

@ pointers to ascii global variables
\end{alltt}

\newpage

\begin{center}
(blank page)
\end{center}

\newpage

\section*{Standard: Assembly Programming III}

On the following page is an Assembly program for a number guessing game. The program picks a ``random'' number between 0 and 127. The user has ten chances to guess it. After each wrong guess, the program tells if it was too high or too low. It looks like this:

\begin{alltt}
    > ./guessing_game
    I'm thinking of a number 0 to 127
    Guess #1: 23
    Too low!
    Guess #2: 78
    Too low!
    Guess #3: 105
    Too high!
    Guess #4: 87
    Too low!
    Guess #5: 93
    Too low!
    Guess #6: 95
    Correct!
\end{alltt}

Ten of the lines in the program have mistakes. Find them and fix them. 

A memory diagram is not required but you may find it helpful.

\vfill

\rule[1ex]{\textwidth}{.1pt}

$\square$ \textbf{P}: You have demonstrated proficiency. Full credit. Well done!

$\square$ \textbf{R}: You are close! Half credit. Submit a revision for full credit

$\square$ \textbf{S}: Partial proficiency. Half credit. Try again on the exam

$\square$ \textbf{I}: You have not yet demonstrated proficiency. Try again on the exam

\newpage

\begin{alltt}
    .section .rodata
intro: .ascii "I'm thinking of a number 0 to 127{\textbackslash}n{\textbackslash}0"
prompt: .ascii "Guess #%d: {\textbackslash}0"
input_format: .ascii "%d{\textbackslash}0"
too_high_reply: .ascii "Too high!{\textbackslash}n{\textbackslash}0"
too_low_reply: .ascii "Too low!{\textbackslash}n{\textbackslash}0"
correct_reply: .ascii "Correct!{\textbackslash}n{\textbackslash}0"
no_more_tries: .ascii "No more tries. Better luck next time!{\textbackslash}n{\textbackslash}0"

    .text
    .global main
main: 
    @ stack frame setup
    push \{fp, lr\}
    add fp, sp, #4
    sub sp, sp, #4
    @ use current time to generate "random" number in r4
    sub r0, fp, #8      @ no errors
    bl time             @ in these
    ldr r4, [fp, #-8]   @ four lines
    and r4, r4, #0x7f   @ i promise
    @ init guess counter
    mov r5, #0
    @ print game intro
    ldr r0, intro_ptr
    bl printf
begin_loop:
    @ increment guess counter, break if max
    add r5, r5, #1
    ifeq r5, #10
    b break_failure
    @ print the numbered prompt
    mov r1, r5
    ldr r0, prompt_ptr
    bl printf
    @ read in the guess
    sub r1, fp, #8
    ldr r0, input_format_ptr
    bl scanf
    ldr r6, [fp, #-8]
    @ compare the guess to the solution
    ifeq r6, r4
    b break_success
    ifgt r6, r4
    b too_low
    b too_high
    @ handle cases
too_low:
    ldr r0, too_low_reply_ptr
    bl printf
    b return
too_high:
    ldr r0, too_high_reply_ptr
    bl printf
    b return
break_success:
    ldr r0, correct_reply_ptr
    bl printf
    b return
break_failure:
    ldr r0, no_more_tries_ptr
    bl printf
    b begin_loop
return:
    @ stack frame teardown
    add sp, sp, #4
    pop \{fp, lr\}
    bx lr

intro_ptr: .word intro
prompt_ptr: .word prompt
input_format_ptr: .word input_format
too_high_reply_ptr: .word too_high_reply
too_low_reply_ptr: .word too_low_reply
correct_reply_ptr: .word correct_reply
no_more_tries_ptr: .word no_more_tries
\end{alltt}


\newpage

\begin{center}
(blank page)
\end{center}


\end{document}
