\documentclass[12pt]{article}
\usepackage[utf8]{inputenc}
\usepackage{amssymb, amsmath,graphicx,hyperref,xcolor}

\setlength{\parindent}{0in}
\setlength{\parskip}{1em}

\usepackage{fancyhdr}
\rhead{}

\pagestyle{fancy}
\lhead{St Olaf College}
\chead{CS 241}
\rhead{Spring 2024}

\cfoot{\thepage}

\begin{document}

Name: \makebox[3in]{\hrulefill
%NAME
\hrulefill}

\vfill

\begin{center}
{\huge Exam 2}
\end{center}

\begin{itemize}
    \item This is a closed-book exam. Please do not use outside resources such as notes, past work, or other people. 
    \item This exam should be doable in under an hour. If you need a little bit of extra time, that's fine. Please don't spend all day on it. 
    \item The classroom is available at the regular time. You are also welcome to choose a different time/place that works for you. 
    \item Once done, scan in your work and email it to me at \texttt{fyfe1@stolaf.edu}. Work must be turned in \underline{by Wednesday night}.
    \item If you have already demonstrated proficiency on a standard, you can skip it here. If you don't remember, check your grade on Moodle.
    \item I have confidence in you!
\end{itemize}

\vfill

I pledge my honor that on this examination I have neither given nor received assistance not explicitly approved by the professor and that I have seen no dishonest work 

\hfill Signed: \makebox[3in]{\hrulefill}

$\square$\quad I have intentionally not signed the pledge. (check only if appropriate)
\newpage

\section*{Assembly Programming I}

%STANDARD_AP1

The following page has the skeleton of an Assembly program with comments. The program does the following:
\begin{itemize}
    \item Create global variables \texttt{n\_koalas} and \texttt{n\_thumbs}
    \item Use \texttt{printf} to ask the user how many koalas they have
    \item Use \texttt{scanf} to accept a value and store it to \texttt{n\_koalas}
    \item Multiply the value by six and store it in \texttt{n\_thumbs}
    \item Use \texttt{printf} to report the total number of thumbs
\end{itemize}

For example:
\begin{verbatim}
    How many koalas do you have? 4
    That's a total of 24 thumbs!
\end{verbatim}

Please fill in the missing code

Note: koalas have two thumbs on each hand, plus one thumb per foot. Six thumbs per koala.

\vfill

\rule[1ex]{\textwidth}{.1pt}

$\square$ \textbf{P}: You have demonstrated proficiency. Full credit. Well done!

$\square$ \textbf{R}: You are close! Half credit. Submit a revision for full credit

$\square$ \textbf{S}: Partial proficiency. Half credit. Try again on the final

$\square$ \textbf{I}: You have not yet demonstrated proficiency. Try again on the final

\newpage



\begin{verbatim}
@ global constants



@ global variables



@ main


    
    @ stack frame setup, no local variables



    @ print the prompt



    @ read the value n_koalas



    @ compute n_thumbs and store the value



    @ load n_thumbs and print the reply



    @ stack frame teardown, return zero



@ pointers
\end{verbatim}
\newpage


\section*{Memory Diagrams}

%STANDARD_MR

The following page has an Assembly program. It does the following:
\begin{itemize}
    \item Allocate two local variables: \texttt{x} and \texttt{x\_squared}
    \item Store a value to \texttt{x}
    \item Send the \underline{addresses} of both variables to the function \texttt{square\_by\_address}
    \item The function loads \texttt{x}, computes its square, and stores it in \texttt{x\_squared}
    \item Back in main, load the values and print them
\end{itemize}

The output looks like:
\begin{verbatim}
    17 squared is 289
\end{verbatim}

Please draw a memory diagram showing the values in registers and main memory at the indicated line. In cases where you don't know the literal value, write what it is --- for example, ``old lr". 

\vfill

\rule[1ex]{\textwidth}{.1pt}

$\square$ \textbf{P}: You have demonstrated proficiency. Full credit. Well done!

$\square$ \textbf{R}: You are close! Half credit. Submit a revision for full credit

$\square$ \textbf{S}: Partial proficiency. Half credit. Try again on the exam

$\square$ \textbf{I}: You have not yet demonstrated proficiency. Try again on the exam

\newpage

\begin{verbatim}
    .section .rodata
output: .ascii "%d squared is %d\n\0"

    .text
square_by_address:
    push {fp, lr}
    add fp, sp, #4
    @ load value from input address
    ldr r0, [r0]
    @ square it and store the result
    mul r0, r0, r0
    str r0, [r1]
    mov r0, #0
    @               STOP HERE FOR MEMORY DIAGRAM
    pop {fp, pc}

    .global main
main: 
    push {fp, lr}
    add fp, sp, #4
    sub sp, sp, #8
    @ fp-8 is x, fp-12 is x_squared
    mov r0, #17
    str r0, [fp, #-8]
    @ pass the addresses to the function
    sub r0, fp, #8
    sub r1, fp, #12
    bl square_by_address
    @ load the values and print them
    ldr r1, [fp, #-8]
    ldr r2, [fp, #-12]
    ldr r0, output_ptr
    bl printf
    mov r0, #0
    sub sp, fp, #4
    pop {fp, pc}

output_ptr: .word output
\end{verbatim}

\newpage

\section*{Assembly Programming II}

%STANDARD_AP2

The following page has the skeleton of an Assembly program with comments. The program does the following:
\begin{itemize}
    \item Allocate local variables \texttt{x}, \texttt{y}, \texttt{z}, and \texttt{xyz}.
    \item Store initial values \texttt{x = 5}, \texttt{y = 7}, and \texttt{z = 11}
    \item Pass the values \texttt{x}, \texttt{y}, and \texttt{z} to the function \texttt{multiply\_three}
    \item Within the function, calculate \texttt{x*y*z} and return it
    \item Back in \texttt{main}, store the result to \texttt{xyz}
\end{itemize}

Please fill in the missing code

\vfill

\rule[1ex]{\textwidth}{.1pt}

$\square$ \textbf{P}: You have demonstrated proficiency. Full credit. Well done!

$\square$ \textbf{R}: You are close! Half credit. Submit a revision for full credit

$\square$ \textbf{S}: Partial proficiency. Half credit. Try again on the final

$\square$ \textbf{I}: You have not yet demonstrated proficiency. Try again on the final

\newpage

\begin{verbatim}
@ function multiply_three



    @ return x*y*z



@ function main


    
    @ stack frame setup, four local variables



    @ store initial values for x, y, and z



    @ call multiply_three



    @ store return value as xyz



    @ stack frame teardown, return 0



\end{verbatim}

\newpage

\section*{Standard: Assembly Programming III}

%STANDARD_AP3

The following page has the skeleton of an Assembly program with comments. The program does the following:
\begin{itemize}
    \item Initialize the sum to \texttt{0}
    \item Prompt the user for a number, read it in, and add it to the sum
    \item If the sum is less than 100, loop again
    \item Otherwise, exit the loop and print the sum
\end{itemize}

For example:
\begin{verbatim}
    Please give a number: 12
    Please give a number: 45
    Please give a number: 3
    Please give a number: 0
    Please give a number: 97
    The sum of all those numbers is: 157
\end{verbatim}

Please fill in the missing code

\vfill

\rule[1ex]{\textwidth}{.1pt}

$\square$ \textbf{P}: You have demonstrated proficiency. Full credit. Well done!

$\square$ \textbf{R}: You are close! Half credit. Submit a revision for full credit

$\square$ \textbf{S}: Partial proficiency. Half credit. Try again on the final

$\square$ \textbf{I}: You have not yet demonstrated proficiency. Try again on the final

\newpage

\begin{verbatim}
@ global constants



@ function main

    
    
    @ stack frame setup



    @ print the prompt



    @ read in the value



    @ maybe repeat the loop

    
    
    @ print the result



    @ stack frame teardown, return 0



@ pointers
\end{verbatim}

\end{document}
