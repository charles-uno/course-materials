\documentclass[12pt]{article}
\usepackage[utf8]{inputenc}
\usepackage{alltt, amssymb, amsmath,graphicx,hyperref,xcolor}

\setlength{\parindent}{0in}
\setlength{\parskip}{1em}

\usepackage{fancyhdr}
\rhead{}

\pagestyle{fancy}
\lhead{St Olaf College}
\chead{CS 241}
\rhead{Spring 2024}

\cfoot{\thepage}

\begin{document}

Name: \makebox[3in]{\hrulefill}

\vfill

\begin{center}
{\huge Quiz 4}
\end{center}

\begin{itemize}
    \item The standards AP2 and AP3 do not cover global variables or terminal input/output. You will \underline{not} be penalized for syntax errors on those.
    \item Please make sure your work is legible. There is a blank page following each exercise. More paper is available if needed.
    \item I have confidence in you!
\end{itemize}

\vfill

I pledge my honor that on this examination I have neither given nor received assistance not explicitly approved by the professor and that I have seen no dishonest work 

\hfill Signed: \makebox[3in]{\hrulefill}

$\square$\quad I have intentionally not signed the pledge. (check only if appropriate)
\newpage

\section*{Standard: Assembly Programming II}

On the next page is a commented skeleton for an Assembly program. It does the following:
\begin{itemize}
    \item Set an initial value \texttt{x} and store it to a local variable.
    \item Call the function \texttt{plus\_one} to compute \texttt{x+1}. Store the result.
    \item Load the values from memory and print them. For example:
    \begin{alltt}
    4 + 1 = 5
    \end{alltt}
\end{itemize}

Please fill in the missing code. 

A memory diagram is not required, but you may find it helpful.

\vfill

\rule[1ex]{\textwidth}{.1pt}

$\square$ \textbf{P}: You have demonstrated proficiency. Full credit. Well done!

$\square$ \textbf{R}: You are close! Half credit. Submit a revision for full credit

$\square$ \textbf{S}: Partial proficiency. Half credit. Try again on the exam

$\square$ \textbf{I}: You have not yet demonstrated proficiency. Try again on the exam

\newpage

\begin{alltt}
    .section .rodata
plus_one_output: .ascii "%d + 1 = %d{\textbackslash}n{\textbackslash}0"

    .text
plus_one:
    @ input is x. return x+1


    .global main
main: 
    @ stack frame setup, local variables


    @ set initial value, store it


    @ call plus_one, store result


    @ load both values and print them


    @ stack frame teardown, return 0


plus_one_output_ptr: .word plus_one_output
\end{alltt}

\newpage

\begin{center}
(blank page)
\end{center}

\newpage

\section*{Standard: Assembly Programming III}

On the next page is a commented skeleton for an Assembly program. It does the following:
\begin{itemize}
    \item Prompt the user to enter the number \texttt{5}.
    \item Check if they entered the right number.
    \item If not, ask them again (and again, and again, etc). For example:
    \begin{alltt}
    Please enter 5: 4
    Incorrect!
    Please enter 5: 12
    Incorrect!
    Please enter 5: 5
    Correct!
    \end{alltt}
\end{itemize}

Please fill in the missing code. 

There are several correct ways to write the solution. You do not need to follow the comments exactly.


\vfill

\rule[1ex]{\textwidth}{.1pt}

$\square$ \textbf{P}: You have demonstrated proficiency. Full credit. Well done!

$\square$ \textbf{R}: You are close! Half credit. Submit a revision for full credit

$\square$ \textbf{S}: Partial proficiency. Half credit. Try again on the exam

$\square$ \textbf{I}: You have not yet demonstrated proficiency. Try again on the exam

\newpage

\begin{alltt}
    .section .rodata
prompt: .ascii "Please enter 5: {\textbackslash}0"
input_format: .ascii "%d{\textbackslash}0"
reply_incorrect: .ascii "Incorrect!{\textbackslash}n{\textbackslash}0"
reply_correct: .ascii "Correct!{\textbackslash}n{\textbackslash}0"

    .text
    .global main
main: 
    @ stack frame setup


    @ print the prompt
    ldr r0, prompt_ptr
    bl printf
    @ read in the guess
    sub r1, fp, #8
    ldr r0, input_format_ptr
    bl scanf
    ldr r2, [fp, #-8]
    @ compare the guess to the solution


    @ maybe repeat the loop


    @ maybe break


    @ stack frame teardown, return 0


prompt_ptr: .word prompt
input_format_ptr: .word input_format
reply_correct_ptr: .word reply_correct
reply_incorrect_ptr: .word reply_incorrect

\end{alltt}


\newpage

\begin{center}
(blank page)
\end{center}


\end{document}
