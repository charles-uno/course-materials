

\Section{Electrical Circuits}


\begin{frame}{The Setup}
	\begin{tikzpicture}
		% Paths, nodes and wires:
		\node[ground] at (2.571, 5.659){};
		\node[sground, yscale=-1] at (2.571, 8.484){};
		\node[shape=rectangle, minimum width=0.608cm, minimum height=0.536cm] at (1.857, 9){} node[anchor=north west, align=left, text width=0.22cm, inner sep=6pt] at (1.536, 9.286){3V};
		\node[shape=rectangle, minimum width=0.608cm, minimum height=0.536cm] at (1.893, 5.286){} node[anchor=north west, align=left, text width=0.22cm, inner sep=6pt] at (1.571, 5.571){0V};
		\node[shape=rectangle, minimum width=0.608cm, minimum height=0.536cm] at (0.536, 7.143){} node[anchor=north west, align=left, text width=0.22cm, inner sep=6pt] at (0.214, 7.429){input};
		\node[shape=rectangle, minimum width=0.608cm, minimum height=0.536cm] at (2.429, 7.143){} node[anchor=north west, align=left, text width=0.22cm, inner sep=6pt] at (2.107, 7.429){logic};
		\node[shape=rectangle, minimum width=0.608cm, minimum height=0.536cm] at (3.821, 7.143){} node[anchor=north west, align=left, text width=0.22cm, inner sep=6pt] at (3.5, 7.429){output};
	\end{tikzpicture}
\end{frame}



\Subsection{Resistors}



\begin{frame}{What is a resistor?}
	light bulb

	ceramic

	the coils in your toaster

	something to soak up energy

	if you run electric current through it and it gets hot, it's a resistor (or at least has resistance)
\end{frame}


\begin{frame}{Voltage Drops in the Resistor}
	\begin{tikzpicture}
		% Paths, nodes and wires:
		\node[ground] at (2.571, 5.659){};
		\draw (2.571, 8.484) to[american resistor, /tikz/circuitikz/bipoles/length=1.12cm] (2.571, 7.199);
		\node[sground, yscale=-1] at (2.571, 8.484){};
		\node[shape=rectangle, minimum width=0.608cm, minimum height=0.536cm] at (1.964, 8.571){} node[anchor=north west, align=left, text width=0.22cm, inner sep=6pt] at (1.643, 8.857){3V};
		\node[shape=rectangle, minimum width=0.608cm, minimum height=0.536cm] at (1.107, 6.143){} node[anchor=north west, align=left, text width=0.22cm, inner sep=6pt] at (0.786, 6.429){0V};
		\draw[-latex] (1.571, 6.286) -- (2.286, 6.857);
		\draw[-latex] (1.571, 6) -- (2.286, 5.571);
		\draw (2.571, 7.199) -- (2.571, 5.659);
	\end{tikzpicture}
\end{frame}




\begin{frame}{Voltage only drops if it has somewhere to go}
	\begin{tikzpicture}
		% Paths, nodes and wires:
		\node[ground] at (2.571, 5.659){};
		\draw (2.571, 8.484) to[american resistor, /tikz/circuitikz/bipoles/length=1.12cm] (2.571, 7.199);
		\node[sground, yscale=-1] at (2.571, 8.484){};
		\node[shape=rectangle, minimum width=0.608cm, minimum height=0.536cm] at (0.893, 7.857){} node[anchor=north west, align=left, text width=0.22cm, inner sep=6pt] at (0.571, 8.143){3V};
		\node[shape=rectangle, minimum width=0.608cm, minimum height=0.536cm] at (0.857, 5.429){} node[anchor=north west, align=left, text width=0.22cm, inner sep=6pt] at (0.536, 5.714){0V};
		\draw[-latex] (1.286, 8.143) -- (2.286, 8.571);
		\draw[-latex] (1.429, 7.714) -- (2.286, 7.286);
		\draw[-latex] (1.571, 5.429) -- (2.286, 5.571);
	\end{tikzpicture}
\end{frame}


\begin{frame}{NOT ALLOWED}


	\begin{columns}
		\begin{column}{0.5\textwidth}
			We assume:
			\begin{itemize}
				\item Top rail fixed at 12V
				\item Bottom rail fixed at 0V
				\item Wires have zero resistance
			\end{itemize}
		\end{column}
		\begin{column}{0.5\textwidth}
			\begin{tikzpicture}
				% Paths, nodes and wires:
				\node[ground] at (2.571, 5.659){};
				\node[sground, yscale=-1] at (2.571, 8.484){};
				\draw (2.571, 8.484) -- (2.571, 5.659);
			\end{tikzpicture}		\end{column}
	\end{columns}

	If you try to build this, one of those assumptions will fail




\end{frame}


\Subsection{Transistors}

\begin{frame}{Historical Transistors}
	\includegraphics[width=0.5\columnwidth]{images/tube-transistor-wiki}
\end{frame}


\begin{frame}{Modern Transistors}
	\includegraphics[width=\columnwidth]{images/semiconductor-transistor-overkill}
\end{frame}



\begin{frame}{Transistor Behavior}

	https://www.101computing.net/creating-logic-gates-using-transistors/
	
	\begin{tikzpicture}
		% Paths, nodes and wires:
		\draw (0, 5.429) -- (4, 5.429);
		\node[ground] at (4, 5.429){};
		\draw (0, 8) -- (4, 8);
		\node[shape=rectangle, minimum width=0.893cm, minimum height=0.536cm] at (0.321, 7.857){} node[anchor=north west, align=left, text width=0.505cm, inner sep=6pt] at (-0.143, 8.143){12V};
		\node[shape=rectangle, minimum width=0.893cm, minimum height=0.536cm] at (0.286, 5.714){} node[anchor=north west, align=left, text width=0.505cm, inner sep=6pt] at (-0.179, 6){0V};
		\node[npn] at (2.571, 6.429){};
		\node[shape=rectangle, minimum width=0.608cm, minimum height=0.536cm] at (0.143, 6.714){} node[anchor=north west, align=left, text width=0.22cm, inner sep=6pt] at (-0.179, 7){A};
		\draw (1.714, 6.429) -- (0, 6.429);
		\draw (2.571, 7.199) -- (2.571, 8);
		\draw (2.571, 5.659) -- (2.571, 5.429);
	\end{tikzpicture}
\end{frame}


\begin{frame}{Transistor Behavior}
\end{frame}


\begin{frame}{Transistor Behavior}
\end{frame}



\begin{frame}{Example Logic}
	\begin{tikzpicture}
		% Paths, nodes and wires:
		\draw (0, 4) -- (4, 4);
		\node[ground] at (4, 4){};
		\draw (2.84, 7.484) to[american resistor] (2.857, 6);
		\draw (0, 8) -- (4, 8);
		\node[shape=rectangle, minimum width=1.75cm, minimum height=0.679cm] at (0.679, 7.643){} node[anchor=north west, align=left, text width=1.362cm, inner sep=6pt] at (-0.214, 8){12 Volts};
		\node[shape=rectangle, minimum width=1.536cm, minimum height=0.679cm] at (0.643, 4.357){} node[anchor=north west, align=left, text width=1.148cm, inner sep=6pt] at (-0.143, 4.714){0 Volts};
		\node[npn] at (2.84, 5){};
		\draw (2.84, 7.484) -- (2.857, 8);
		\draw (2.84, 4.23) -- (2.857, 4);
		\draw (2, 5) -- (0, 5);
		\draw (2.857, 6) -- (2.84, 5.77);
		\draw (2.857, 5.857) -- (4, 5.857);
		\node[shape=rectangle, minimum width=0.715cm, minimum height=0.679cm] at (0.232, 5.214){} node[anchor=north west, align=left, text width=0.327cm, inner sep=6pt] at (-0.143, 5.571){A};
		\node[shape=rectangle, minimum width=0.715cm, minimum height=0.679cm] at (3.857, 6.214){} node[anchor=north west, align=left, text width=0.327cm, inner sep=6pt] at (3.482, 6.571){$\overline{A}$};
	\end{tikzpicture}
\end{frame}



