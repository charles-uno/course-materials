
\Section{Syllabus}

\begin{frame}{Course Overview}

	\includegraphics[width=\columnwidth]{images/dune-robot-hbo}

	{\center
		Thou shalt not make a machine in the likeless of a human mind
	}

	{\small \hfill
		The Orange Catholic Bible, Dune, Frank Herbert
	}
\end{frame}


\begin{frame}{Course Overview}

	Computers feel like magic. You type words. The machine does complex things. Especially in the era of large language models

	But it's not magic. Computers are made of semiconductors, metal, and magnets. Three different kinds of rocks

	This course is about demystifying the machine and exploring how, fundamentally, computers do what they do. Specifically, we'll be looking at:
	\begin{itemize}
		\item The fundamental components of a computer
		\item How data is stored on a computer
		\item Building electrical circuits to perform logic
		\item Programming in Assembly (a very low-level language)
	\end{itemize}
\end{frame}

\Subsection{Grading}

\begin{frame}{Course Grade Breakdown}
	\begin{itemize}
		\item Quizzes and final: 45\%
		\item Homework and labs: 45\%
		\item Participation: 10\%
	\end{itemize}
\end{frame}

\begin{frame}{Standards-Based Grading}
	\begin{itemize}
		\item There are nine standards in the class. Each is worth 5\% of your grade.
		\item There are three quizzes. Each quiz covers three standards.
		\item The final covers all nine standards.
		\item If you demonstrate proficiency on the quiz, you're done with that standard. Full credit. You can skip that part of the final.
		\item If you demonstrate partial proficiency on the quiz, you get half credit. Try again on the final for full credit.
		\item If you do not demonstrate proficiency on the quiz, you get no credit. You can still get full credit on the final!
	\end{itemize}
\end{frame}


\begin{frame}{Homework and Labs}
	\begin{itemize}
		\item One or more lab exercises for each standard. We start these together in class. You may need to finish on your own time.
		\item We will have a homework assignment for each standard.
		\item Please work together in small groups.
		\item Please ensure that the work you turn in reflects your own understanding.
	\end{itemize}
\end{frame}

\begin{frame}{Late Work Policy}
	Please try to get your work in on time.
	\begin{itemize}
		\item If you don't get your work done on time, you may fall behind in the class.
		\item Late work is inconvenient for the TAs. They will be grumpy and vindictive when they grade your work.
		\item TAs will have their own finals to worry about! Late work at the end of the semester might be refused.
	\end{itemize}
	If you need an extension, please reach out beforehand.
\end{frame}

\begin{frame}{Peer Reviews}
	\begin{itemize}
		\item Participation score (10\% of your total grade) is mostly based on peer reviews.
		\item Reviews will be short. Less than one page.
		\item You can get full credit here without too much trouble. Show up. Work together. Make it easy for your peers to say nice things about you.
		\item I recommend keeping notes over the course of the semester when someone is particularly helpful, insightful, etc. Good reviews have concrete examples.
	\end{itemize}
\end{frame}

\Subsection{Office Hours}

\begin{frame}{Office Hours}
	TODO
\end{frame}


\Subsection{Disability Accommodations}

\begin{frame}{Disability Accommodations}
	I am committed to supporting the learning of all students in my class. If you have already
	registered with Disability and Access (DAC) and have your letter of accommodations, please
	meet with me as soon as possible to discuss, plan, and implement your accommodations in the
	course. If you have or think you have a disability (learning, sensory, physical, chronic health,
	mental health or attentional), please contact Disability and Access staff at 507-786-3288 or by visiting \href{https://wp.stolaf.edu/academic-support/dac}{their website}.


	If you have an accommodation that allows you extra time or a low distraction environment for
	quizzes and exams, please email me at least three days before each quiz or exam, so that I can
	make sure to reserve a room.
\end{frame}


\Subsection{Religious Accommodations}

\begin{frame}{Religious Accommodations}
	As part of my commitment to make St Olaf an inclusive community, I will provide students with
	reasonable religious accommodations. If you will be missing class for a religious observance or
	require another religious accommodation, please meet with me to discuss these.
\end{frame}


\Subsection{Academic Integrity}

\begin{frame}{Academic Integrity}

	Plagiarism is a serious academic offense. Hand in your own work. Give credit appropriately when you draw from the work of others. For more information please see:

	\begin{itemize}
		\item \href{https://wp.stolaf.edu/facultyhandbook/academic-integrity-faculty-handbook-category-2}{Faculty Handbook: Academic Integrity}
		\item \href{https://wp.stolaf.edu/honorcouncil/}{The Honor Code}
		\item \href{https://wp.stolaf.edu/roadmap-to-academic-integrity}{Roadmap to Academic Integrity}
	\end{itemize}

	Work that violates this policy will typically receive no credit. In especially serious cases the penalty can be an F in the course.
\end{frame}

\Subsection{AI Use Policy}

\begin{frame}{AI Usage Policy}
	Please not submit AI work as your own.

	You may use AI tools when studying, but be careful. These tools are very good with mainstream languages like Python. We're working with Assembly, which is very much \textbf{not} a mainstream language.

\end{frame}


\Subsection{St Olaf Land Acknowledgement}

\begin{frame}{St Olaf Land Acknowledgement}
	We stand on the homelands of the Wahpekute Band of the Dakota Nation. We honor with
	gratitude the people who have stewarded the land throughout the generations and their ongoing
	contributions to this region. We acknowledge the ongoing injustices that we have committed
	against the Dakota Nation, and we wish to interrupt this legacy, beginning with acts of healing
	and honest storytelling about this place.

	For more information about land acknowledgement statements at St Olaf, please see \href{https://wp.stolaf.edu/education/land-acknowledgement/}{here} and \href{https://wp.stolaf.edu/equity-inclusion/land-acknowledgement/}{here}.
\end{frame}


\Subsection{Statement of Inclusivity}

\begin{frame}{Statement of Inclusivity}
	In keeping with St. Olaf College's mission statement, this class strives to be an inclusive
	learning community, respecting those of differing backgrounds and beliefs. As a community, we
	aim to be respectful to all citizens in this class, regardless of race, ethnicity, religion, gender or
	sexual orientation.
\end{frame}


\Subsection{Gender Pronouns}

\begin{frame}{Gender Pronouns}
	This course affirms people of all gender expressions and gender identities. If you go by a
	different name than what is on the class roster, please let me know.

	Using correct gender
	pronouns is important to me, so you are encouraged to share your pronouns with me and
	correct me if a mistake is made. If you have any questions or concerns, please do not hesitate
	to contact me.
\end{frame}


\Subsection{Multilingual Student Support}

\begin{frame}{Multilingual Student Support}
	I am committed to making course content accessible to all students. If English is not your first
	language and this causes you concern about the course, please speak with me. Students who
	would like extra support with writing or speaking in English can also contact the language support specialist (\href{mailto:berryag@stolaf.edu}{berryag@stolaf.edu}) in the Academic Success Center.
\end{frame}


\Subsection{Mental Health}

\begin{frame}{Mental Health}

	I greatly value your experience in this class, and it is my duty to facilitate a safe, caring, and
	productive learning environment.

	I recognize that you may experience a range of emotional, physical, and/or psychological
	issues, both in and out of the classroom, that may distract you from your learning.

	If you are experiencing such issues, please do not hesitate to come see me. I am here to listen.
	We can also discuss what further resources might be available to you.
\end{frame}

\Subsection{Required Referrals}


\begin{frame}{Required Referrals}
	You are welcome to talk to me about circumstances outside the course that
	affect your classroom experience or acadmic performance. However, please keep
	in mind that I am required to refer cases of discrimination, harassment,
	sexual misconduct, and violence.

	Here are some resources where you can share privately:

	\begin{itemize}
		\item Boe House Counseling Center
		\item College Pastors and Chaplains
		\item Sexual Assault Resource Network (SARN)
		\item Health Services
		\item TimelyCare
	\end{itemize}
\end{frame}




\Subsection{St. Olaf Pride Statement}

\begin{frame}{St. Olaf Pride Statement}
	As an Ole, I will practice:
	\begin{itemize}
		\item Passion for learning and pursuit of vocation
		\item Respect for the worth and dignity of all people
		\item Integrity at all times, in all circumstances
		\item Dedication to a life of service, and
		\item Engagement with my community and the world.
	\end{itemize}
\end{frame}









\Subsection{Important Links}

\begin{frame}{Important Links}
	\begin{itemize}
		\item ARM tutorial https://computerscience.chemeketa.edu/armTutorial/index.html
		\item ARM simulator https://cpulator.01xz.net/?sys=arm
		\item circuit simulator: https://circuitverse.org/simulator
	\end{itemize}
\end{frame}



