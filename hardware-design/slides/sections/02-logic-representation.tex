
\Section{Boolean Logic}

\Subsection{Truth Tables}

\begin{frame}{Ones and Zeroes}
	\begin{itemize}
		\item Boolean logic uses two values: true and false.
		\item On a computer, true is 1 and false is 0.
		\item A single boolean value is simple. When we put multiple boolean values together, complex logic becomes possible.
	\end{itemize}
\end{frame}

\begin{frame}{Truth Tables}
	\begin{tabular}{c c c}
		A & B & A or B \\
		\hline
		1 & 1 & 1      \\
		1 & 0 & 1      \\
		0 & 1 & 1      \\
		0 & 0 & 0
	\end{tabular}
\end{frame}

\begin{frame}{Truth Tables}
	\begin{tabular}{c c c}
		A & B & (A or B) and \big((not A) or (not B)\big) \\
		\hline
		1 & 1 & 0                                         \\
		1 & 0 & 1                                         \\
		0 & 1 & 1                                         \\
		0 & 0 & 0
	\end{tabular}
\end{frame}

\Subsection{Logic Gates}

\begin{frame}{Logic Gates: NOT}
	We use an overline to indicate NOT. So $\overline{A}$ means "not A".

	\begin{columns}
		\begin{column}{0.5\textwidth}
			\begin{tikzpicture}
				% Paths, nodes and wires:
				\node[ieeestd not port] at (12.877, 10.375){};
				\node[ocirc] at (12, 10.375){};
				\node[ocirc] at (13.75, 10.375){};
				\node[shape=rectangle, minimum width=0.354cm, minimum height=0.59cm] at (11.555, 10.437){} node[anchor=north west, align=left, text width=-0.034cm, inner sep=6pt] at (11.36, 10.75){A};
				\node[shape=rectangle, minimum width=0.744cm, minimum height=0.59cm] at (14.36, 10.563){} node[anchor=north west, align=left, text width=0.356cm, inner sep=6pt] at (13.971, 10.875){$\overline{A}$};
			\end{tikzpicture}
		\end{column}
		\begin{column}{0.5\textwidth}
			\begin{center}
				\begin{tabular}{c c}
					A & $\overline{A}$ \\
					\hline
					1 & 0              \\
					0 & 1
				\end{tabular}
			\end{center}
		\end{column}
	\end{columns}

	You may also see $\neg A$ or $\sim A$
\end{frame}


\begin{frame}{Logic Gates: AND, NAND}
	We use a dot to indicate AND. So $A \cdot B$ means "A and B". This is pretty easy to remember since the dot is also used for multiplication

	NAND is its opposite. "A nand B" is the same as "not (A and B)"

	\begin{columns}
		\begin{column}{0.5\textwidth}
			\begin{tikzpicture}
				% Paths, nodes and wires:
				\node[ieeestd and port] at (1.11, 10.22){};
				\node[ieeestd nand port] at (1.081, 7.47){};
				\node[shape=rectangle, minimum width=0.215cm, minimum height=0.59cm] at (-0.375, 10.563){} node[anchor=north west, align=left, text width=-0.173cm, inner sep=6pt] at (-0.5, 10.875){A};
				\node[shape=rectangle, minimum width=0.215cm, minimum height=0.465cm] at (-0.375, 10){} node[anchor=north west, align=left, text width=-0.173cm, inner sep=6pt] at (-0.5, 10.25){B};
				\node[shape=rectangle, minimum width=0.215cm, minimum height=0.59cm] at (-0.404, 7.813){} node[anchor=north west, align=left, text width=-0.173cm, inner sep=6pt] at (-0.529, 8.125){A};
				\node[shape=rectangle, minimum width=0.215cm, minimum height=0.465cm] at (-0.404, 7.25){} node[anchor=north west, align=left, text width=-0.173cm, inner sep=6pt] at (-0.529, 7.5){B};
				\node[shape=rectangle, minimum width=0.744cm, minimum height=0.59cm] at (2.61, 10.25){} node[anchor=north west, align=left, text width=0.356cm, inner sep=6pt] at (2.221, 10.563){${A \cdot B}$};
				\node[shape=rectangle, minimum width=0.744cm, minimum height=0.59cm] at (2.581, 7.563){} node[anchor=north west, align=left, text width=0.356cm, inner sep=6pt] at (2.192, 7.875){$\overline{A \cdot B}$};
			\end{tikzpicture}
		\end{column}
		\begin{column}{0.5\textwidth}
			\begin{center}
				\begin{tabular}{c c c c}
					A & B & $A \cdot B$ & $\overline{A \cdot B}$ \\
					\hline
					1 & 1 & 1           & 0                      \\
					1 & 0 & 0           & 1                      \\
					0 & 1 & 0           & 1                      \\
					0 & 0 & 0           & 1
				\end{tabular}
			\end{center}
		\end{column}
	\end{columns}

	AND can also be indicated by $A \land B$ or $A \& B$
\end{frame}


\begin{frame}{Logic Gates: OR, NOR}
	We use a plus sign for OR. So $A+B$ means "A or B".

	NOR is its opposite

	\begin{columns}
		\begin{column}{0.5\textwidth}
			\begin{tikzpicture}
				% Paths, nodes and wires:
				\node[ieeestd or port] at (1.11, 10.22){};
				\node[ieeestd nor port] at (1.081, 7.47){};
				\node[shape=rectangle, minimum width=0.215cm, minimum height=0.59cm] at (-0.375, 10.563){} node[anchor=north west, align=left, text width=-0.173cm, inner sep=6pt] at (-0.5, 10.875){A};
				\node[shape=rectangle, minimum width=0.215cm, minimum height=0.465cm] at (-0.375, 10){} node[anchor=north west, align=left, text width=-0.173cm, inner sep=6pt] at (-0.5, 10.25){B};
				\node[shape=rectangle, minimum width=0.215cm, minimum height=0.59cm] at (-0.404, 7.813){} node[anchor=north west, align=left, text width=-0.173cm, inner sep=6pt] at (-0.529, 8.125){A};
				\node[shape=rectangle, minimum width=0.215cm, minimum height=0.465cm] at (-0.404, 7.25){} node[anchor=north west, align=left, text width=-0.173cm, inner sep=6pt] at (-0.529, 7.5){B};
				\node[shape=rectangle, minimum width=0.744cm, minimum height=0.59cm] at (2.61, 10.25){} node[anchor=north west, align=left, text width=0.356cm, inner sep=6pt] at (2.221, 10.563){${A + B}$};
				\node[shape=rectangle, minimum width=0.744cm, minimum height=0.59cm] at (2.581, 7.563){} node[anchor=north west, align=left, text width=0.356cm, inner sep=6pt] at (2.192, 7.875){$\overline{A + B}$};
			\end{tikzpicture}
		\end{column}
		\begin{column}{0.5\textwidth}
			\begin{center}
				\begin{tabular}{c c c c}
					A & B & $A + B$ & $\overline{A + B}$ \\
					\hline
					1 & 1 & 1       & 0                  \\
					1 & 0 & 1       & 0                  \\
					0 & 1 & 1       & 0                  \\
					0 & 0 & 0       & 1
				\end{tabular}
			\end{center}
		\end{column}
	\end{columns}

	You may also see $A \lor B$ or $A | B$
\end{frame}


\begin{frame}{Logic Gates: XOR, XNOR}
	XOR (pronounced "exor" or "zor") means exclusive OR. So A is true or B is true, but not both. It's indicated by $\text{A}\oplus\text{B}$

	XNOR (pronounced like "exnor" or "snore")
	\begin{columns}
		\begin{column}{0.5\textwidth}
			\begin{tikzpicture}
				% Paths, nodes and wires:
				\node[ieeestd xor port] at (1.11, 10.22){};
				\node[ieeestd xnor port] at (1.081, 7.47){};
				\node[shape=rectangle, minimum width=0.215cm, minimum height=0.59cm] at (-0.375, 10.563){} node[anchor=north west, align=left, text width=-0.173cm, inner sep=6pt] at (-0.5, 10.875){A};
				\node[shape=rectangle, minimum width=0.215cm, minimum height=0.465cm] at (-0.375, 10){} node[anchor=north west, align=left, text width=-0.173cm, inner sep=6pt] at (-0.5, 10.25){B};
				\node[shape=rectangle, minimum width=0.215cm, minimum height=0.59cm] at (-0.404, 7.813){} node[anchor=north west, align=left, text width=-0.173cm, inner sep=6pt] at (-0.529, 8.125){A};
				\node[shape=rectangle, minimum width=0.215cm, minimum height=0.465cm] at (-0.404, 7.25){} node[anchor=north west, align=left, text width=-0.173cm, inner sep=6pt] at (-0.529, 7.5){B};
				\node[shape=rectangle, minimum width=0.744cm, minimum height=0.59cm] at (2.61, 10.25){} node[anchor=north west, align=left, text width=0.356cm, inner sep=6pt] at (2.221, 10.563){${\text{A}\oplus\text{B}}$};
				\node[shape=rectangle, minimum width=0.744cm, minimum height=0.59cm] at (2.581, 7.563){} node[anchor=north west, align=left, text width=0.356cm, inner sep=6pt] at (2.192, 7.875){$\overline{\text{A}\oplus\text{B}}$};
			\end{tikzpicture}
		\end{column}
		\begin{column}{0.5\textwidth}
			\begin{center}
				\begin{tabular}{c c c c}
					A & B & ${A \oplus B}$ & $\overline{A \oplus B}$ \\
					\hline
					1 & 1 & 0              & 1                       \\
					1 & 0 & 1              & 0                       \\
					0 & 1 & 1              & 0                       \\
					0 & 0 & 0              & 1
				\end{tabular}
			\end{center}
		\end{column}
	\end{columns}

	You may also see $A \veebar B$ or $A \not\equiv B$
\end{frame}


\begin{frame}{Exercises}
	\begin{enumerate}
		\item TODO
	\end{enumerate}
\end{frame}


\Subsection{Logic Circuits}

\begin{frame}{Multiple Gates Together}
	We can chain multiple gates together to perform complex operations. This is called a logic circuit

	\begin{tikzpicture}
		% Paths, nodes and wires:
		\node[shape=rectangle, minimum width=0.59cm, minimum height=0.59cm] at (1.571, 3.973){} node[anchor=north west, align=left, text width=0.202cm, inner sep=6pt] at (1.259, 4.286){B};
		\node[ieeestd and port] at (7.205, 5.286){};
		\node[shape=rectangle, minimum width=0.59cm, minimum height=0.59cm] at (1.545, 4.741){} node[anchor=north west, align=left, text width=0.202cm, inner sep=6pt] at (1.232, 5.054){A};
		\node[ieeestd or port] at (4.919, 6.143){};
		\node[ieeestd nand port] at (4.919, 4.429){};
		\draw (6, 6.143) -- (6, 5.571) -- (6.143, 5.571);
		\draw (6, 4.429) -- (6, 5) -- (6.123, 5.006);
		\node[jump crossing] at (3.429, 4.717){};
		\draw (3.838, 4.709) -- (3.571, 4.714);
		\draw (3.838, 4.149) -- (2, 4.143);
		\draw (3.429, 4.577) -- (3.429, 4.143);
		\draw (3.429, 4.857) -- (3.429, 5.857) -- (3.838, 5.863);
		\draw (3.286, 4.714) -- (2, 4.714);
		\draw (3.838, 6.423) -- (2.857, 6.429) -- (2.857, 4.714);
		\node[shape=rectangle, minimum width=0.59cm, minimum height=0.59cm] at (8.741, 5.312){} node[anchor=north west, align=left, text width=0.202cm, inner sep=6pt] at (8.429, 5.625){??};
	\end{tikzpicture}

	What does this one do?
\end{frame}

\begin{frame}{Understanding a Logic Circuit}
	We can work from left to right, making note of what each gate does

	\begin{tikzpicture}
		% Paths, nodes and wires:
		\node[shape=rectangle, minimum width=0.59cm, minimum height=0.59cm] at (1.571, 3.973){} node[anchor=north west, align=left, text width=0.202cm, inner sep=6pt] at (1.259, 4.286){B};
		\node[ieeestd and port] at (7.205, 5.286){};
		\node[shape=rectangle, minimum width=0.59cm, minimum height=0.59cm] at (1.545, 4.741){} node[anchor=north west, align=left, text width=0.202cm, inner sep=6pt] at (1.232, 5.054){A};
		\node[ieeestd or port] at (4.919, 6.143){};
		\node[ieeestd nand port] at (4.919, 4.429){};
		\draw (6, 6.143) -- (6, 5.571) -- (6.143, 5.571);
		\draw (6, 4.429) -- (6, 5) -- (6.123, 5.006);
		\node[jump crossing] at (3.429, 4.717){};
		\draw (3.838, 4.709) -- (3.571, 4.714);
		\draw (3.838, 4.149) -- (2, 4.143);
		\draw (3.429, 4.577) -- (3.429, 4.143);
		\draw (3.429, 4.857) -- (3.429, 5.857) -- (3.838, 5.863);
		\draw (3.286, 4.714) -- (2, 4.714);
		\draw (3.838, 6.423) -- (2.857, 6.429) -- (2.857, 4.714);
		\node[shape=rectangle, minimum width=1.447cm, minimum height=0.59cm] at (6, 6.714){} node[anchor=north west, align=left, text width=1.059cm, inner sep=6pt] at (5.259, 7.027){A+B};
		\node[shape=rectangle, minimum width=1.25cm, minimum height=0.59cm] at (6.357, 4.143){} node[anchor=north west, align=left, text width=0.862cm, inner sep=6pt] at (5.714, 4.455){$\overline{A \cdot B}$};
		\node[shape=rectangle, minimum width=3.108cm, minimum height=0.59cm] at (9.714, 5.402){} node[anchor=north west, align=left, text width=2.72cm, inner sep=6pt] at (8.143, 5.714){??};
	\end{tikzpicture}
\end{frame}




\begin{frame}{Understanding a Logic Circuit}
	We can work from left to right, making note of what each gate does

	\begin{tikzpicture}
		% Paths, nodes and wires:
		\node[shape=rectangle, minimum width=0.59cm, minimum height=0.59cm] at (1.571, 3.973){} node[anchor=north west, align=left, text width=0.202cm, inner sep=6pt] at (1.259, 4.286){B};
		\node[ieeestd and port] at (7.205, 5.286){};
		\node[shape=rectangle, minimum width=0.59cm, minimum height=0.59cm] at (1.545, 4.741){} node[anchor=north west, align=left, text width=0.202cm, inner sep=6pt] at (1.232, 5.054){A};
		\node[ieeestd or port] at (4.919, 6.143){};
		\node[ieeestd nand port] at (4.919, 4.429){};
		\draw (6, 6.143) -- (6, 5.571) -- (6.143, 5.571);
		\draw (6, 4.429) -- (6, 5) -- (6.123, 5.006);
		\node[jump crossing] at (3.429, 4.717){};
		\draw (3.838, 4.709) -- (3.571, 4.714);
		\draw (3.838, 4.149) -- (2, 4.143);
		\draw (3.429, 4.577) -- (3.429, 4.143);
		\draw (3.429, 4.857) -- (3.429, 5.857) -- (3.838, 5.863);
		\draw (3.286, 4.714) -- (2, 4.714);
		\draw (3.838, 6.423) -- (2.857, 6.429) -- (2.857, 4.714);
		\node[shape=rectangle, minimum width=1.447cm, minimum height=0.59cm] at (6, 6.714){} node[anchor=north west, align=left, text width=1.059cm, inner sep=6pt] at (5.259, 7.027){A+B};
		\node[shape=rectangle, minimum width=1.25cm, minimum height=0.59cm] at (6.357, 4.143){} node[anchor=north west, align=left, text width=0.862cm, inner sep=6pt] at (5.714, 4.455){$\overline{A \cdot B}$};
		\node[shape=rectangle, minimum width=3.108cm, minimum height=0.59cm] at (9.714, 5.402){} node[anchor=north west, align=left, text width=2.72cm, inner sep=6pt] at (8.143, 5.714){$(A+B) \cdot  (\overline{A \cdot B})$};
	\end{tikzpicture}
\end{frame}


\begin{frame}{Understanding a Logic Circuit}
	Then we can write the corresponding truth table

	\begin{tikzpicture}
		% Paths, nodes and wires:
		\node[shape=rectangle, minimum width=0.59cm, minimum height=0.59cm] at (1.571, 3.973){} node[anchor=north west, align=left, text width=0.202cm, inner sep=6pt] at (1.259, 4.286){B};
		\node[ieeestd and port] at (7.205, 5.286){};
		\node[shape=rectangle, minimum width=0.59cm, minimum height=0.59cm] at (1.545, 4.741){} node[anchor=north west, align=left, text width=0.202cm, inner sep=6pt] at (1.232, 5.054){A};
		\node[ieeestd or port] at (4.919, 6.143){};
		\node[ieeestd nand port] at (4.919, 4.429){};
		\draw (6, 6.143) -- (6, 5.571) -- (6.143, 5.571);
		\draw (6, 4.429) -- (6, 5) -- (6.123, 5.006);
		\node[jump crossing] at (3.429, 4.717){};
		\draw (3.838, 4.709) -- (3.571, 4.714);
		\draw (3.838, 4.149) -- (2, 4.143);
		\draw (3.429, 4.577) -- (3.429, 4.143);
		\draw (3.429, 4.857) -- (3.429, 5.857) -- (3.838, 5.863);
		\draw (3.286, 4.714) -- (2, 4.714);
		\draw (3.838, 6.423) -- (2.857, 6.429) -- (2.857, 4.714);
		\node[shape=rectangle, minimum width=1.447cm, minimum height=0.59cm] at (6, 6.714){} node[anchor=north west, align=left, text width=1.059cm, inner sep=6pt] at (5.259, 7.027){A+B};
		\node[shape=rectangle, minimum width=1.25cm, minimum height=0.59cm] at (6.357, 4.143){} node[anchor=north west, align=left, text width=0.862cm, inner sep=6pt] at (5.714, 4.455){$\overline{A \cdot B}$};
		\node[shape=rectangle, minimum width=3.108cm, minimum height=0.59cm] at (9.714, 5.402){} node[anchor=north west, align=left, text width=2.72cm, inner sep=6pt] at (8.143, 5.714){$(A+B) \cdot  (\overline{A \cdot B})$};
	\end{tikzpicture}

	\begin{center}
		\begin{tabular}{c c c c c}
			A & B & ${A + B}$ & $\overline{A \cdot B}$ & $(A+B) \cdot  (\overline{A \cdot B})$ \\
			\hline
			1 & 1 & 1         & 0                      & 0                                     \\
			1 & 0 & 1         & 1                      & 1                                     \\
			0 & 1 & 1         & 1                      & 1                                     \\
			0 & 0 & 0         & 1                      & 0
		\end{tabular}
	\end{center}
\end{frame}



\begin{frame}{Drawing a Logic Expression}
	Similarly, if we have a logical expression, we can draw it as a diagram from right to left.

	\begin{tikzpicture}
		% Paths, nodes and wires:
		\node[ieeestd and port] at (7.205, 5.286){};
		\node[shape=rectangle, minimum width=1.447cm, minimum height=0.59cm] at (5.598, 5.741){} node[anchor=north west, align=left, text width=1.059cm, inner sep=6pt] at (4.857, 6.054){A+B};
		\node[shape=rectangle, minimum width=1.25cm, minimum height=0.59cm] at (5.5, 4.884){} node[anchor=north west, align=left, text width=0.862cm, inner sep=6pt] at (4.857, 5.196){$\overline{A \cdot B}$};
		\node[shape=rectangle, minimum width=3.108cm, minimum height=0.59cm] at (9.714, 5.402){} node[anchor=north west, align=left, text width=2.72cm, inner sep=6pt] at (8.143, 5.714){$(A+B) \cdot  (\overline{A \cdot B})$};
	\end{tikzpicture}
\end{frame}


\begin{frame}{Drawing a Logic Expression}
	\begin{tikzpicture}
		% Paths, nodes and wires:
		\node[shape=rectangle, minimum width=0.59cm, minimum height=0.59cm] at (3.402, 5.741){} node[anchor=north west, align=left, text width=0.202cm, inner sep=6pt] at (3.089, 6.054){B};
		\node[ieeestd and port] at (7.205, 5.286){};
		\node[shape=rectangle, minimum width=0.59cm, minimum height=0.59cm] at (3.375, 6.509){} node[anchor=north west, align=left, text width=0.202cm, inner sep=6pt] at (3.062, 6.821){A};
		\node[ieeestd or port] at (4.919, 6.143){};
		\draw (6, 6.143) -- (6, 5.571) -- (6.143, 5.571);
		\node[shape=rectangle, minimum width=1.447cm, minimum height=0.59cm] at (6, 6.714){} node[anchor=north west, align=left, text width=1.059cm, inner sep=6pt] at (5.259, 7.027){A+B};
		\node[shape=rectangle, minimum width=1.25cm, minimum height=0.59cm] at (5.357, 4.857){} node[anchor=north west, align=left, text width=0.862cm, inner sep=6pt] at (4.714, 5.17){$\overline{A \cdot B}$};
		\node[shape=rectangle, minimum width=3.108cm, minimum height=0.59cm] at (9.714, 5.402){} node[anchor=north west, align=left, text width=2.72cm, inner sep=6pt] at (8.143, 5.714){$(A+B) \cdot  (\overline{A \cdot B})$};
	\end{tikzpicture}
\end{frame}


\begin{frame}{Drawing a Logic Expression}
	\begin{tikzpicture}
		% Paths, nodes and wires:
		\node[shape=rectangle, minimum width=0.59cm, minimum height=0.59cm] at (3.402, 5.741){} node[anchor=north west, align=left, text width=0.202cm, inner sep=6pt] at (3.089, 6.054){B};
		\node[ieeestd and port] at (7.205, 5.286){};
		\node[shape=rectangle, minimum width=0.59cm, minimum height=0.59cm] at (3.375, 6.509){} node[anchor=north west, align=left, text width=0.202cm, inner sep=6pt] at (3.062, 6.821){A};
		\node[ieeestd or port] at (4.919, 6.143){};
		\node[ieeestd nand port] at (4.919, 4.429){};
		\draw (6, 6.143) -- (6, 5.571) -- (6.143, 5.571);
		\draw (6, 4.429) -- (6, 5) -- (6.123, 5.006);
		\node[shape=rectangle, minimum width=1.447cm, minimum height=0.59cm] at (6, 6.714){} node[anchor=north west, align=left, text width=1.059cm, inner sep=6pt] at (5.259, 7.027){A+B};
		\node[shape=rectangle, minimum width=1.25cm, minimum height=0.59cm] at (6.357, 4.143){} node[anchor=north west, align=left, text width=0.862cm, inner sep=6pt] at (5.714, 4.455){$\overline{A \cdot B}$};
		\node[shape=rectangle, minimum width=3.108cm, minimum height=0.59cm] at (9.714, 5.402){} node[anchor=north west, align=left, text width=2.72cm, inner sep=6pt] at (8.143, 5.714){$(A+B) \cdot  (\overline{A \cdot B})$};
		\node[shape=rectangle, minimum width=0.59cm, minimum height=0.59cm] at (3.402, 3.973){} node[anchor=north west, align=left, text width=0.202cm, inner sep=6pt] at (3.089, 4.286){B};
		\node[shape=rectangle, minimum width=0.59cm, minimum height=0.59cm] at (3.375, 4.741){} node[anchor=north west, align=left, text width=0.202cm, inner sep=6pt] at (3.062, 5.054){A};
	\end{tikzpicture}
\end{frame}

\begin{frame}{Drawing a Logic Expression}
	\begin{tikzpicture}
		% Paths, nodes and wires:
		\node[shape=rectangle, minimum width=0.59cm, minimum height=0.59cm] at (1.571, 3.973){} node[anchor=north west, align=left, text width=0.202cm, inner sep=6pt] at (1.259, 4.286){B};
		\node[ieeestd and port] at (7.205, 5.286){};
		\node[shape=rectangle, minimum width=0.59cm, minimum height=0.59cm] at (1.545, 4.741){} node[anchor=north west, align=left, text width=0.202cm, inner sep=6pt] at (1.232, 5.054){A};
		\node[ieeestd or port] at (4.919, 6.143){};
		\node[ieeestd nand port] at (4.919, 4.429){};
		\draw (6, 6.143) -- (6, 5.571) -- (6.143, 5.571);
		\draw (6, 4.429) -- (6, 5) -- (6.123, 5.006);
		\node[jump crossing] at (3.429, 4.717){};
		\draw (3.838, 4.709) -- (3.571, 4.714);
		\draw (3.838, 4.149) -- (2, 4.143);
		\draw (3.429, 4.577) -- (3.429, 4.143);
		\draw (3.429, 4.857) -- (3.429, 5.857) -- (3.838, 5.863);
		\draw (3.286, 4.714) -- (2, 4.714);
		\draw (3.838, 6.423) -- (2.857, 6.429) -- (2.857, 4.714);
		\node[shape=rectangle, minimum width=1.447cm, minimum height=0.59cm] at (6, 6.714){} node[anchor=north west, align=left, text width=1.059cm, inner sep=6pt] at (5.259, 7.027){A+B};
		\node[shape=rectangle, minimum width=1.25cm, minimum height=0.59cm] at (6.357, 4.143){} node[anchor=north west, align=left, text width=0.862cm, inner sep=6pt] at (5.714, 4.455){$\overline{A \cdot B}$};
		\node[shape=rectangle, minimum width=3.108cm, minimum height=0.59cm] at (9.714, 5.402){} node[anchor=north west, align=left, text width=2.72cm, inner sep=6pt] at (8.143, 5.714){$(A+B) \cdot  (\overline{A \cdot B})$};
	\end{tikzpicture}
\end{frame}


\begin{frame}{Functional Completeness}
	\begin{columns}
		\begin{column}{0.5\textwidth}
			\begin{itemize}
				\item There are many ways to build gates from other gates.
				\item It's even possible to build all gates using only NAND. This is called functional completeness.
				\item Why might functional completeness be useful?
			\end{itemize}
		\end{column}
		\begin{column}{0.5\textwidth}
			\includegraphics[width=\columnwidth]{images/functional-completeness.png}
		\end{column}
	\end{columns}
\end{frame}


\Subsection{Arithmetic with Circuits}


\begin{frame}{Arithmetic with Circuits}
	\begin{itemize}
		\item Let's say A and B are a pair of one-bit binary numbers
		\item Can we build a circuit to add them?
		\item Let's start with a truth table
	\end{itemize}
\end{frame}


\begin{frame}{Arithmetic with Circuits}
	The four possible cases are:
	\begin{itemize}
		\item $0b0 + 0b0 = 0b0$
		\item $0b1 + 0b0 = 0b1$
		\item $0b0 + 0b1 = 0b1$
		\item $0b1 + 0b1 = 0b10$
	\end{itemize}

	Which we can write as a truth table:
	\begin{center}
		\begin{tabular}{c c c c}
			A & B & C & D \\
			\hline
			1 & 1 & 0 & 0 \\
			1 & 0 & 1 & 0 \\
			0 & 1 & 1 & 0 \\
			0 & 0 & 0 & 1
		\end{tabular}
	\end{center}

	Where C and D are the digits of the result
\end{frame}




\begin{frame}{Arithmetic with Circuits}
	The four possible cases are:
	\begin{itemize}
		\item $0b0 + 0b0 = 0b0$
		\item $0b1 + 0b0 = 0b1$
		\item $0b0 + 0b1 = 0b1$
		\item $0b1 + 0b1 = 0b10$
	\end{itemize}

	Which we can write as a truth table:
	\begin{center}
		\begin{tabular}{c c c c}
			A & B & C & D \\
			\hline
			1 & 1 & 0 & 0 \\
			1 & 0 & 1 & 0 \\
			0 & 1 & 1 & 0 \\
			0 & 0 & 0 & 1
		\end{tabular}
	\end{center}

What is the logical expression for C? How about D?

Let's draw the expression

\end{frame}







\begin{frame}{Multiple Outputs}
	\begin{tikzpicture}
		% Paths, nodes and wires:
		\node[ieeestd and port] at (5.205, 5){};
		\node[shape=rectangle, minimum width=0.59cm, minimum height=0.59cm] at (2.571, 3.688){} node[anchor=north west, align=left, text width=0.202cm, inner sep=6pt] at (2.259, 4){A};
		\node[shape=rectangle, minimum width=0.59cm, minimum height=0.465cm] at (2.598, 2.964){} node[anchor=north west, align=left, text width=0.202cm, inner sep=6pt] at (2.286, 3.214){B};
		\node[ieeestd xor port] at (5.205, 3.286){};
		\node[jump crossing] at (3.854, 3.571){};
		\draw (3.994, 3.571) -- (4.123, 3.566);
		\draw (4.123, 3.006) -- (3, 3);
		\draw (3.714, 3.571) -- (3, 3.571);
		\draw (4.123, 4.72) -- (3.857, 4.714) -- (3.854, 3.711);
		\draw (3.857, 3.429) -- (3.857, 3);
		\draw (4.123, 5.28) -- (3.429, 5.286) -- (3.429, 3.571);
		\node[shape=rectangle, minimum width=0.59cm, minimum height=0.59cm] at (6.688, 5.027){} node[anchor=north west, align=left, text width=0.202cm, inner sep=6pt] at (6.375, 5.339){C};
		\node[shape=rectangle, minimum width=0.59cm, minimum height=0.59cm] at (6.688, 3.286){} node[anchor=north west, align=left, text width=0.202cm, inner sep=6pt] at (6.375, 3.598){D};
	\end{tikzpicture}
\end{frame}



\begin{frame}{Bigger Adder with Carry}
	\begin{tikzpicture}
		% Paths, nodes and wires:
		\node[ieeestd and port] at (5.205, 5){};
		\node[shape=rectangle, minimum width=0.59cm, minimum height=0.59cm] at (2.571, 3.688){} node[anchor=north west, align=left, text width=0.202cm, inner sep=6pt] at (2.259, 4){B};
		\node[shape=rectangle, minimum width=0.59cm, minimum height=0.465cm] at (2.598, 2.964){} node[anchor=north west, align=left, text width=0.202cm, inner sep=6pt] at (2.286, 3.214){C};
		\node[ieeestd xor port] at (5.205, 3.286){};
		\node[jump crossing] at (3.854, 3.571){};
		\draw (3.994, 3.571) -- (4.123, 3.566);
		\draw (4.123, 3.006) -- (3, 3);
		\draw (3.714, 3.571) -- (3, 3.571);
		\draw (4.123, 4.72) -- (3.857, 4.714) -- (3.854, 3.711);
		\draw (3.857, 3.429) -- (3.857, 3);
		\draw (4.123, 5.28) -- (3.429, 5.286) -- (3.429, 3.571);
		\node[ieeestd xor port] at (7.776, 6){};
		\node[ieeestd and port] at (7.776, 7.714){};
		\node[shape=rectangle, minimum width=0.59cm, minimum height=0.59cm] at (2.598, 6.402){} node[anchor=north west, align=left, text width=0.202cm, inner sep=6pt] at (2.286, 6.714){A};
		\node[jump crossing] at (6.429, 6.283){};
		\draw (6.571, 6.286) -- (6.714, 6.286);
		\draw (6.695, 7.434) -- (6.429, 7.429) -- (6.429, 6.429);
		\draw (6.695, 5.72) -- (6.429, 5.714);
		\draw (6.429, 6.143) -- (6.429, 5.714) -- (6.429, 5) -- (6.286, 5);
		\draw (6.695, 7.994) -- (6, 8) -- (6, 6.286) -- (6.286, 6.286);
		\draw (6, 6.286) -- (3, 6.286);
		\draw (6.286, 3.286) -- (8.857, 3.286);
		\node[shape=rectangle, minimum width=0.59cm, minimum height=0.59cm] at (9.402, 7.741){} node[anchor=north west, align=left, text width=0.202cm, inner sep=6pt] at (9.089, 8.054){D};
		\node[shape=rectangle, minimum width=0.59cm, minimum height=0.59cm] at (9.429, 6.027){} node[anchor=north west, align=left, text width=0.202cm, inner sep=6pt] at (9.116, 6.339){E};
		\node[shape=rectangle, minimum width=0.59cm, minimum height=0.59cm] at (9.402, 3.402){} node[anchor=north west, align=left, text width=0.202cm, inner sep=6pt] at (9.089, 3.714){F};
	\end{tikzpicture}
\end{frame}




\begin{frame}{Exercises}
	\begin{enumerate}
		\item Draw a logic circuit for the boolean function
		      $f(A, B) = (A \oplus B) + (A \cdot B)$
		\item Draw a logic circuit for the boolean function
		      $f(A, B) = (A xor B) nor (A and B)$
		\item Write out the truth table for the boolean function
		      $f(A, B) = (A + B) \cdot (\overline{A \cdot B})$. Do you recognize it?
	\end{enumerate}
\end{frame}





\begin{frame}{Control Circuits}

	multiplexers
\end{frame}


