
\Section{Logic Representation}

\Subsection{Boolean Logic}

\begin{frame}{Ones and Zeroes}

	Boolean logic uses two values: true and false.

	On a computer, true is 1 and false is 0.

	A single boolean value is simple. When we put multiple boolean values together, complex logic becomes possible.

\end{frame}

\begin{frame}{Truth Tables}
	\begin{tabular}{c c c}
		A & B & A or B \\
		\hline
		1 & 1 & 1      \\
		1 & 0 & 1      \\
		0 & 1 & 1      \\
		0 & 0 & 0
	\end{tabular}
\end{frame}

\begin{frame}{Truth Tables}
	\begin{tabular}{c c c}
		A & B & (A or B) and \big((not A) or (not B)\big) \\
		\hline
		1 & 1 & 0                                         \\
		1 & 0 & 1                                         \\
		0 & 1 & 1                                         \\
		0 & 0 & 0
	\end{tabular}
\end{frame}

\Subsection{Logic Gates}

\begin{frame}{Logic Gates: NOT}

	We use an overbar to indicate NOT. So $\overline{A}$ means "not A".

	\begin{columns}
		\begin{column}{0.5\textwidth}
			\begin{tikzpicture}
				% Paths, nodes and wires:
				\node[ieeestd not port] at (12.877, 10.375){};
				\node[ocirc] at (12, 10.375){};
				\node[ocirc] at (13.75, 10.375){};
				\node[shape=rectangle, minimum width=0.354cm, minimum height=0.59cm] at (11.555, 10.437){} node[anchor=north west, align=left, text width=-0.034cm, inner sep=6pt] at (11.36, 10.75){A};
				\node[shape=rectangle, minimum width=0.744cm, minimum height=0.59cm] at (14.36, 10.563){} node[anchor=north west, align=left, text width=0.356cm, inner sep=6pt] at (13.971, 10.875){$\overline{A}$};
			\end{tikzpicture}
		\end{column}
		\begin{column}{0.5\textwidth}
			\begin{center}
				\begin{tabular}{c c}
					A & $\overline{A}$ \\
					\hline
					1 & 0                     \\
					1 & 0                     \\
					0 & 1                     \\
					0 & 1
				\end{tabular}
			\end{center}
		\end{column}
	\end{columns}

	You may also see $\neg A$ or $\sim A$
\end{frame}



\begin{frame}{Logic Gates: AND, NAND}

	We use a dot to indicate AND. So $A \cdot B$ means "A and B". This is pretty easy to remember since the dot is also used for multiplication

	NAND is its opposite. "A nand B" is the same as "not (A and B)"

	\begin{columns}
		\begin{column}{0.5\textwidth}
			\begin{tikzpicture}
				% Paths, nodes and wires:
				\node[ieeestd and port] at (1.11, 10.22){};
				\node[ieeestd nand port] at (1.081, 7.47){};
				\node[shape=rectangle, minimum width=0.215cm, minimum height=0.59cm] at (-0.375, 10.563){} node[anchor=north west, align=left, text width=-0.173cm, inner sep=6pt] at (-0.5, 10.875){A};
				\node[shape=rectangle, minimum width=0.215cm, minimum height=0.465cm] at (-0.375, 10){} node[anchor=north west, align=left, text width=-0.173cm, inner sep=6pt] at (-0.5, 10.25){B};
				\node[shape=rectangle, minimum width=0.215cm, minimum height=0.59cm] at (-0.404, 7.813){} node[anchor=north west, align=left, text width=-0.173cm, inner sep=6pt] at (-0.529, 8.125){A};
				\node[shape=rectangle, minimum width=0.215cm, minimum height=0.465cm] at (-0.404, 7.25){} node[anchor=north west, align=left, text width=-0.173cm, inner sep=6pt] at (-0.529, 7.5){B};
				\node[shape=rectangle, minimum width=0.744cm, minimum height=0.59cm] at (2.61, 10.25){} node[anchor=north west, align=left, text width=0.356cm, inner sep=6pt] at (2.221, 10.563){${A \cdot B}$};
				\node[shape=rectangle, minimum width=0.744cm, minimum height=0.59cm] at (2.581, 7.563){} node[anchor=north west, align=left, text width=0.356cm, inner sep=6pt] at (2.192, 7.875){$\overline{A \cdot B}$};
			\end{tikzpicture}
		\end{column}
		\begin{column}{0.5\textwidth}
			\begin{center}
				\begin{tabular}{c c c c}
					A & B & $A \cdot B$ & $\overline{A \cdot B}$ \\
					\hline
					1 & 1 & 1                       & 0                                  \\
					1 & 0 & 0                       & 1                                  \\
					0 & 1 & 0                       & 1                                  \\
					0 & 0 & 0                       & 1
				\end{tabular}
			\end{center}
		\end{column}
	\end{columns}



	AND can also be indicated by $A \land B$ or $A \& B$
\end{frame}




\begin{frame}{Logic Gates: OR, NOR}

	We use a plus sign for OR. So $A+B$ means "A or B".

	NOR is its opposite

	\begin{columns}
		\begin{column}{0.5\textwidth}
			\begin{tikzpicture}
				% Paths, nodes and wires:
				\node[ieeestd or port] at (1.11, 10.22){};
				\node[ieeestd nor port] at (1.081, 7.47){};
				\node[shape=rectangle, minimum width=0.215cm, minimum height=0.59cm] at (-0.375, 10.563){} node[anchor=north west, align=left, text width=-0.173cm, inner sep=6pt] at (-0.5, 10.875){A};
				\node[shape=rectangle, minimum width=0.215cm, minimum height=0.465cm] at (-0.375, 10){} node[anchor=north west, align=left, text width=-0.173cm, inner sep=6pt] at (-0.5, 10.25){B};
				\node[shape=rectangle, minimum width=0.215cm, minimum height=0.59cm] at (-0.404, 7.813){} node[anchor=north west, align=left, text width=-0.173cm, inner sep=6pt] at (-0.529, 8.125){A};
				\node[shape=rectangle, minimum width=0.215cm, minimum height=0.465cm] at (-0.404, 7.25){} node[anchor=north west, align=left, text width=-0.173cm, inner sep=6pt] at (-0.529, 7.5){B};
				\node[shape=rectangle, minimum width=0.744cm, minimum height=0.59cm] at (2.61, 10.25){} node[anchor=north west, align=left, text width=0.356cm, inner sep=6pt] at (2.221, 10.563){${A + B}$};
				\node[shape=rectangle, minimum width=0.744cm, minimum height=0.59cm] at (2.581, 7.563){} node[anchor=north west, align=left, text width=0.356cm, inner sep=6pt] at (2.192, 7.875){$\overline{A + B}$};
			\end{tikzpicture}
		\end{column}
		\begin{column}{0.5\textwidth}
			\begin{center}
				\begin{tabular}{c c c c}
					A & B & $A + B$ & $\overline{A + B}$ \\
					\hline
					1 & 1 & 1                   & 0                              \\
					1 & 0 & 1                   & 0                              \\
					0 & 1 & 1                   & 0                              \\
					0 & 0 & 0                   & 1
				\end{tabular}
			\end{center}
		\end{column}
	\end{columns}

	You may also see $A \lor B$ or $A | B$
\end{frame}


\begin{frame}{Logic Gates: XOR, XNOR}
	XOR (pronounced "exor" or "zor") means exclusive OR. So A is true or B is true, but not both. It's indicated by $\text{A}\oplus\text{B}$

	XNOR (pronounced like "exnor" or "snore")
	\begin{columns}
		\begin{column}{0.5\textwidth}
			\begin{tikzpicture}
				% Paths, nodes and wires:
				\node[ieeestd xor port] at (1.11, 10.22){};
				\node[ieeestd xnor port] at (1.081, 7.47){};
				\node[shape=rectangle, minimum width=0.215cm, minimum height=0.59cm] at (-0.375, 10.563){} node[anchor=north west, align=left, text width=-0.173cm, inner sep=6pt] at (-0.5, 10.875){A};
				\node[shape=rectangle, minimum width=0.215cm, minimum height=0.465cm] at (-0.375, 10){} node[anchor=north west, align=left, text width=-0.173cm, inner sep=6pt] at (-0.5, 10.25){B};
				\node[shape=rectangle, minimum width=0.215cm, minimum height=0.59cm] at (-0.404, 7.813){} node[anchor=north west, align=left, text width=-0.173cm, inner sep=6pt] at (-0.529, 8.125){A};
				\node[shape=rectangle, minimum width=0.215cm, minimum height=0.465cm] at (-0.404, 7.25){} node[anchor=north west, align=left, text width=-0.173cm, inner sep=6pt] at (-0.529, 7.5){B};
				\node[shape=rectangle, minimum width=0.744cm, minimum height=0.59cm] at (2.61, 10.25){} node[anchor=north west, align=left, text width=0.356cm, inner sep=6pt] at (2.221, 10.563){${\text{A}\oplus\text{B}}$};
				\node[shape=rectangle, minimum width=0.744cm, minimum height=0.59cm] at (2.581, 7.563){} node[anchor=north west, align=left, text width=0.356cm, inner sep=6pt] at (2.192, 7.875){$\overline{\text{A}\oplus\text{B}}$};
			\end{tikzpicture}
		\end{column}
		\begin{column}{0.5\textwidth}
			\begin{center}
				\begin{tabular}{c c c c}
					A & B & ${A \oplus B}$ & $\overline{A \oplus B}$ \\
					\hline
					1 & 1 & 0                          & 1                                   \\
					1 & 0 & 1                          & 0                                   \\
					0 & 1 & 1                          & 0                                   \\
					0 & 0 & 0                          & 1
				\end{tabular}
			\end{center}
		\end{column}
	\end{columns}

	You may also see $A \veebar B$ or $A \not\equiv B$
\end{frame}


\begin{frame}{Exercises}
	\begin{enumerate}
		\item TODO
	\end{enumerate}
\end{frame}


\Subsection{Logic Circuits}



\begin{frame}{Logic Circuits}
	\begin{tikzpicture}
		% Paths, nodes and wires:
		\node[ieeestd and port] at (5.205, 5){};
		\node[shape=rectangle, minimum width=0.59cm, minimum height=0.59cm] at (-1.884, 5.027){} node[anchor=north west, align=left, text width=0.202cm, inner sep=6pt] at (-2.196, 5.339){A};
		\node[shape=rectangle, minimum width=0.59cm, minimum height=0.465cm] at (-1.857, 3.036){} node[anchor=north west, align=left, text width=0.202cm, inner sep=6pt] at (-2.17, 3.286){B};
		\node[shape=rectangle, minimum width=2.994cm, minimum height=0.965cm] at (7.8, 4.857){} node[anchor=north west, align=left, text width=2.606cm, inner sep=6pt] at (6.286, 5.357){???};
		\node[ieeestd or port] at (2.919, 6){};
		\node[ieeestd not port] at (0.837, 5){};
		\node[ieeestd not port] at (0.837, 3){};
		\node[ieeestd or port] at (2.919, 4){};
		\node[jump crossing] at (-0.429, 4.997){};
		\draw (4, 6) -- (4, 5.286) -- (4.123, 5.28);
		\draw (4, 4) -- (4, 4.714) -- (4.123, 4.72);
		\draw (1.714, 5) -- (1.714, 4.286) -- (1.838, 4.28);
		\draw (1.714, 3) -- (1.714, 3.714) -- (1.838, 3.72);
		\draw (-0.04, 3) -- (-1.571, 3);
		\draw (-0.429, 4.857) -- (-0.429, 3);
		\draw (-0.569, 4.997) -- (-1.571, 5);
		\draw (-0.286, 5) -- (-0.04, 5);
		\draw (1.838, 5.72) -- (-0.429, 5.714) -- (-0.429, 5.143);
		\draw (1.838, 6.28) -- (-1, 6.286) -- (-1, 5);
	\end{tikzpicture}
\end{frame}




\begin{frame}{Logic Circuits}
	\begin{tikzpicture}
		% Paths, nodes and wires:
		\node[ieeestd and port] at (5.205, 5){};
		\node[shape=rectangle, minimum width=0.59cm, minimum height=0.59cm] at (-1.884, 5.027){} node[anchor=north west, align=left, text width=0.202cm, inner sep=6pt] at (-2.196, 5.339){A};
		\node[shape=rectangle, minimum width=0.59cm, minimum height=0.465cm] at (-1.857, 3.036){} node[anchor=north west, align=left, text width=0.202cm, inner sep=6pt] at (-2.17, 3.286){B};
		\node[shape=rectangle, minimum width=2.994cm, minimum height=0.965cm] at (7.771, 5){} node[anchor=north west, align=left, text width=2.606cm, inner sep=6pt] at (6.257, 5.5){$(A+B)\cdot(\overline{A}+\overline{B})$};
		\node[ieeestd or port] at (2.919, 6){};
		\node[ieeestd not port] at (0.837, 5){};
		\node[ieeestd not port] at (0.837, 3){};
		\node[ieeestd or port] at (2.919, 4){};
		\node[shape=rectangle, minimum width=0.59cm, minimum height=0.59cm] at (3.598, 6.545){} node[anchor=north west, align=left, text width=0.202cm, inner sep=6pt] at (3.286, 6.857){A+B};
		\node[shape=rectangle, minimum width=0.59cm, minimum height=0.679cm] at (1.027, 4.571){} node[anchor=north west, align=left, text width=0.202cm, inner sep=6pt] at (0.714, 4.929){$\overline{A}$};
		\node[shape=rectangle, minimum width=0.59cm, minimum height=0.59cm] at (1.17, 2.571){} node[anchor=north west, align=left, text width=0.202cm, inner sep=6pt] at (0.857, 2.884){$\overline{B}$};
		\node[shape=rectangle, minimum width=1.036cm, minimum height=0.965cm] at (3.75, 3.357){} node[anchor=north west, align=left, text width=0.648cm, inner sep=6pt] at (3.214, 3.857){${\overline{A}+\overline{B}}$};
		\node[jump crossing] at (-0.429, 4.997){};
		\draw (4, 6) -- (4, 5.286) -- (4.123, 5.28);
		\draw (4, 4) -- (4, 4.714) -- (4.123, 4.72);
		\draw (1.714, 5) -- (1.714, 4.286) -- (1.838, 4.28);
		\draw (1.714, 3) -- (1.714, 3.714) -- (1.838, 3.72);
		\draw (-0.04, 3) -- (-1.571, 3);
		\draw (-0.429, 4.857) -- (-0.429, 3);
		\draw (-0.569, 4.997) -- (-1.571, 5);
		\draw (-0.286, 5) -- (-0.04, 5);
		\draw (1.838, 5.72) -- (-0.429, 5.714) -- (-0.429, 5.143);
		\draw (1.838, 6.28) -- (-1, 6.286) -- (-1, 5);
	\end{tikzpicture}
\end{frame}





\begin{frame}{Adder with Carry}
	\begin{tikzpicture}
		% Paths, nodes and wires:
		\node[ieeestd and port] at (5.205, 5){};
		\node[shape=rectangle, minimum width=0.59cm, minimum height=0.59cm] at (2.571, 3.688){} node[anchor=north west, align=left, text width=0.202cm, inner sep=6pt] at (2.259, 4){A};
		\node[shape=rectangle, minimum width=0.59cm, minimum height=0.465cm] at (2.598, 2.964){} node[anchor=north west, align=left, text width=0.202cm, inner sep=6pt] at (2.286, 3.214){B};
		\node[ieeestd xor port] at (5.205, 3.286){};
		\node[jump crossing] at (3.854, 3.571){};
		\draw (3.994, 3.571) -- (4.123, 3.566);
		\draw (4.123, 3.006) -- (3, 3);
		\draw (3.714, 3.571) -- (3, 3.571);
		\draw (4.123, 4.72) -- (3.857, 4.714) -- (3.854, 3.711);
		\draw (3.857, 3.429) -- (3.857, 3);
		\draw (4.123, 5.28) -- (3.429, 5.286) -- (3.429, 3.571);
		\node[shape=rectangle, minimum width=0.59cm, minimum height=0.59cm] at (6.688, 5.027){} node[anchor=north west, align=left, text width=0.202cm, inner sep=6pt] at (6.375, 5.339){C};
		\node[shape=rectangle, minimum width=0.59cm, minimum height=0.59cm] at (6.688, 3.286){} node[anchor=north west, align=left, text width=0.202cm, inner sep=6pt] at (6.375, 3.598){D};
	\end{tikzpicture}
\end{frame}



\begin{frame}{Bigger Adder with Carry}
	\begin{tikzpicture}
		% Paths, nodes and wires:
		\node[ieeestd and port] at (5.205, 5){};
		\node[shape=rectangle, minimum width=0.59cm, minimum height=0.59cm] at (2.571, 3.688){} node[anchor=north west, align=left, text width=0.202cm, inner sep=6pt] at (2.259, 4){B};
		\node[shape=rectangle, minimum width=0.59cm, minimum height=0.465cm] at (2.598, 2.964){} node[anchor=north west, align=left, text width=0.202cm, inner sep=6pt] at (2.286, 3.214){C};
		\node[ieeestd xor port] at (5.205, 3.286){};
		\node[jump crossing] at (3.854, 3.571){};
		\draw (3.994, 3.571) -- (4.123, 3.566);
		\draw (4.123, 3.006) -- (3, 3);
		\draw (3.714, 3.571) -- (3, 3.571);
		\draw (4.123, 4.72) -- (3.857, 4.714) -- (3.854, 3.711);
		\draw (3.857, 3.429) -- (3.857, 3);
		\draw (4.123, 5.28) -- (3.429, 5.286) -- (3.429, 3.571);
		\node[ieeestd xor port] at (7.776, 6){};
		\node[ieeestd and port] at (7.776, 7.714){};
		\node[shape=rectangle, minimum width=0.59cm, minimum height=0.59cm] at (2.598, 6.402){} node[anchor=north west, align=left, text width=0.202cm, inner sep=6pt] at (2.286, 6.714){A};
		\node[jump crossing] at (6.429, 6.283){};
		\draw (6.571, 6.286) -- (6.714, 6.286);
		\draw (6.695, 7.434) -- (6.429, 7.429) -- (6.429, 6.429);
		\draw (6.695, 5.72) -- (6.429, 5.714);
		\draw (6.429, 6.143) -- (6.429, 5.714) -- (6.429, 5) -- (6.286, 5);
		\draw (6.695, 7.994) -- (6, 8) -- (6, 6.286) -- (6.286, 6.286);
		\draw (6, 6.286) -- (3, 6.286);
		\draw (6.286, 3.286) -- (8.857, 3.286);
		\node[shape=rectangle, minimum width=0.59cm, minimum height=0.59cm] at (9.402, 7.741){} node[anchor=north west, align=left, text width=0.202cm, inner sep=6pt] at (9.089, 8.054){D};
		\node[shape=rectangle, minimum width=0.59cm, minimum height=0.59cm] at (9.429, 6.027){} node[anchor=north west, align=left, text width=0.202cm, inner sep=6pt] at (9.116, 6.339){E};
		\node[shape=rectangle, minimum width=0.59cm, minimum height=0.59cm] at (9.402, 3.402){} node[anchor=north west, align=left, text width=0.202cm, inner sep=6pt] at (9.089, 3.714){F};
	\end{tikzpicture}
\end{frame}




\begin{frame}{Exercises}

	\begin{enumerate}
		\item Draw a logic circuit for the boolean function
		      $f(A, B) = (A \oplus B) + (A \cdot B)$

		\item Draw a logic circuit for the boolean function
		      $f(A, B) = (A xor B) nor (A and B)$

		\item Write out the truth table for the boolean function
		      $f(A, B) = (A + B) \cdot (\overline{A \cdot B})$. Do you recognize it?

	\end{enumerate}
\end{frame}





\begin{frame}{Control Circuits}

	multiplexers
\end{frame}


