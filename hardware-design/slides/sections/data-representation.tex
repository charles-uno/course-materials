
\Section{Data Representation}

\begin{frame}{How do computers store information?}
Computers work using electrical signals which can be on or off. If we let:
\begin{itemize}
    \item On means 1
    \item Off means 0
\end{itemize}
We can use sequences of 0s and 1s to represent information
\end{frame}

\Subsection{Positive Integers in Binary}

\begin{frame}{Numbers in Base Ten (aka Decimal)}
Let's start by talking about how we write numbers normally. For example, let's look at the number 2051. 
\begin{align*}
2051 &= 2000 + 50 + 1 \\
&= 2 \times 10^3 \;+\; 0 \times 10^2 \;+\; 5 \times 10^1 \;+\; 1 \times 10^0
\end{align*}
This way of writing numbers is called base ten. Each place in the number is scaled by a power of ten, and the digit in that place is less than ten. 
Base ten is convenient for people because we have ten fingers. 
Computers don't have fingers.
\end{frame}
    
\begin{frame}{Numbers in Base Two (aka Binary)}
We always use the prefix ``0b'' to avoid confusion
\end{frame}

\begin{frame}{Addition in Binary}
We'll come back to subtraction later
\end{frame}
    
\begin{frame}{Multiplication in Binary}
We'll come back to division later
\end{frame}


\begin{frame}{Exercises}
\begin{enumerate}
    \item Convert 0b1011 from binary to decimal
\end{enumerate}
\end{frame}

\Subsection{Positive Integers in Hexadecimal}

\Subsection{Negative Numbers}

\Subsection{Fractions and Decimals}

