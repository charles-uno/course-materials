
\Section{Data Representation}

\begin{frame}{How do computers store information?}
Computers work using electrical signals which can be on or off. If we let:
\begin{itemize}
    \item On means 1
    \item Off means 0
\end{itemize}
We can use sequences of 0s and 1s to represent information
\end{frame}

\Subsection{Positive Integers in Binary}

\begin{frame}{Numbers in Base Ten (aka Decimal)}
Let's start by talking about how we write numbers normally. 

For example, let's look at the number 109: 
\begin{align*}
109 &= 100 \;+\; 0 \;+\; 9 \\
&= 1 \times 10^2 \;+\; 0 \times 10^1 \;+\; 9 \times 10^0
\end{align*}
This way of writing numbers is called base ten.
We use ten digits (0 to 9 inclusive) and each position in the number is scaled by a power of ten.

In terms of math, there is nothing special about base ten. 
We probably use it because we have ten fingers. 
Some ancient civilizations used sompletely different systems for counting!
\end{frame}
    
\begin{frame}{Numbers in Base Two (aka Binary)}
Computers don't have fingers. They express everything as sequences of 1s and 0s. This format is called binary:

\begin{align*}
0b1101101 &= 
1 \! \times \! 2^6 +
1 \! \times \! 2^5 +
0 \! \times \! 2^4 +
1 \! \times \! 2^3 +
1 \! \times \! 2^2 +
0 \! \times \! 2^1 +
1 \! \times \! 2^0 \\
&= 64 + 32 + 0 + 8 + 4 + 0 + 1 \\
&= 109 \\
\end{align*}
Importantly: we always use the prefix ``0b'' to avoid confusion when writing numbers in binary.

1101 is one thousand one hundred and one

0b1101 is thirteen
\end{frame}


\begin{frame}{Converting from Decimal to Binary}

If the number is odd, append a 1 on the left. Otherwise, append a 0. Then divide your decimal number by two and ignore any remainder. Repeat until your decimal number is zero. \\

For example, starting with 58:

\begin{itemize}
\item 58 is even, so append \Highlight{0}. Divide by two, leaving 29.
\item 29 is odd, so append \Highlight{1}0. Divide by two, leaving 14. 
\item 14 is odd, so append \Highlight{0}10. Divide by two, leaving 7.
\item 7 is odd, so append \Highlight{1}010. Divide by two, leaving 3.
\item 3 is odd, so append \Highlight{1}1010. Divide by two, leaving 1.
\item 1 is odd, so append \Highlight{1}11010. Divide by two, leaving 0.
\end{itemize}

So 58 in binary is 0b111010.

\end{frame}

\begin{frame}{Converting from Binary to Decimal}

We can confirm by converting back. The rightmost bit is worth 1, then 2, then 4, and so on:

\begin{align*}
    0b111010 &= 
1 \! \times \! 2^5 +
1 \! \times \! 2^4 +
1 \! \times \! 2^3 +
0 \! \times \! 2^2 +
1 \! \times \! 2^1 +
0 \! \times \! 2^0 \\
 &= 32 + 16  + 8 + 0 + 2 + 0\\
 &= 58
\end{align*}

\end{frame}

\begin{frame}{Addition in Binary}
Binary addition works just like decimal addition. 
Start from the right, add straight down, and carry when you run out of digits.

For example:
\[
\begin{array}{ccccccccc}
    1 & 1 & 1 & 1 & 1 &   &   &   &   \\ 
      & 0 & 0 & 1 & 1 & 1 & 0 & 0 & 0 \\ 
    + & 1 & 1 & 1 & 0 & 1 & 0 & 0 & 1 \\
    \hline
    1 & 0 & 0 & 1 & 0 & 0 & 0 & 0 & 1 \\ 
\end{array}
\]

Question: what happens if we try to perform this operation but we only have 8 bits to store the answer? 
\end{frame}

\begin{frame}{Multiplication in Binary}
Binary multiplication works just like decimal multiplication. 
Multiply the top number by the rightmost digit of the bottom number.
Then move to the next line, add a zero, and repeat for the next digit.
Finally, add up the lines. For example:
\[
\begin{array}{cccccc}
        &   &        & 1 & 1 & 0 \\ 
        &   & \times & 1 & 0 & 1 \\
    \hline
        &   &        & 1 & 1 & 0 \\ 
        &   & 0      & 0 & 0 & 0 \\
        + & 1 & 1      & 0 & 0 & 0 \\
        \hline 
        & 1 & 1      & 1 & 1 & 0 \\
\end{array}
\]
\end{frame}


\begin{frame}{Division in Binary}
To divide by 2, shift the bits one place to the right. \\

To divide by 4, shift the bits two places to the right. \\

That's about as deep as we go in this class. If you're curious to learn more, see the textbook.

\end{frame}

\begin{frame}{Sample Exercises}
Make sure to show your work.
\vfill 
\begin{enumerate}
    \item Convert 0b1011 from binary to decimal. 
    \vfill
    \item Convert 47 from decimal to binary.
    \vfill
    \item Add 0b1001 + 0b1011 in binary. Convert to decimal to check your work.
    \vfill
    \item Multiply 0b1101 $\times$ 0b110 in binary. Convert to decimal to verify your work.
    \vfill 
\end{enumerate}
\end{frame}

\Subsection{Negative Numbers}

\begin{frame}{First Attempt: Signed Magnitude}

We can use the first digit to hold the sign, then the rest of the digits to hold magnitue:

    \begin{itemize}
        \item 0b\Highlight{0}1011001 is positive 10110001, so 89
        \item 0b\Highlight{1}1011001 is negative 1011001, so -89 
    \end{itemize}

This is nice and straightforward!

\end{frame}

\begin{frame}{Addition and Subtraction with Signed Magnitude}
    Adding positive numbers works just the same. But what happens if we throw a minus sign in there?
    
    For example, let's look at 12 - 5. First, rewrite it as 12 + (-5). Then: 

    \[
\begin{array}{ccccccccc}
      & 0 & 0 & 0 & 0 & 1 & 1 & 0 & 0 \\ 
    + & 1 & 0 & 0 & 0 & 0 & 1 & 0 & 1 \\
    \hline
      & ? & ? & ? & ? & ? & ? & ? & ? \\ 
\end{array}
\]

We know the result should be 7 (0b00000111). But our regular rules for addition do not get us there.
\end{frame}

\begin{frame}{Zero with Signed Magnitude}

Using signed magnitude, 0b00000000 is zero. 

And 0b10000000 is negative zero (which is also zero).

Using this convention, we have to worry about the difference between numerical equality and bitwise equality. That seems pretty messy.
\end{frame}

\begin{frame}{Can We Do Better?}
Signed magnitude was a swing and a miss. What do we want when we talk about negative numbers?
\begin{itemize}
    \item We want positive numbers to work like we expect
    \item We want 0b00000000 to be zero, with no ambiguity
    \item We want addition and subtraction to work the same for positive and negative
\end{itemize}
\end{frame}

\begin{frame}{Another Idea: Two's Complement}
    We know 1 - 1 = 0. Put another way, 1 + (-1) = 0. Can we work backwards from there to figure out how to write -1?
    \[
        \begin{array}{ccccccccc}
              & 0 & 0 & 0 & 0 & 0 & 0 & 0 & 1 \\
            + & 1 & 1 & 1 & 1 & 1 & 1 & 1 & 1 \\ 
            \hline
              & 0 & 0 & 0 & 0 & 0 & 0 & 0 & 0 \\ 
        \end{array}
        \]

Remember: on a computer, our numbers have to fit in a set number of bits. 
Anything past that gets thrown away. 
This is called overflow.
In this case, overflow works in our favor!
\end{frame}

\begin{frame}{Flipping Signs in Two's Complement}
Working backwards:
    \begin{itemize}
    \item 0b00000000 is 0
    \item 0b11111111 is -1
    \item 0b11111110 is -2
    \item 0b11111101 is -3
    \item 0b11111100 is -4
    \item etc
\end{itemize}

To flip the sign of a two's complement integer, flip all the bits then add 1. Ignore the overflow (if any). This works for both positive and negative numbers. 
\end{frame}

\begin{frame}{Sample Exercises}
Make sure to show your work.
\begin{enumerate}
    \item Convert 0b10010111 from two's complement binary to decimal.
    \item Convert -47 from decimal to two's complement binary. 
    \item Subtract 0b01101001 - 0b10001100. These are two's complement binary numbers. Convert to decimal to check your result.
    \item Add 0b01101001 + 0b10001100. These are two's complement binary numbers. Convert to decimal to check your result.
\end{enumerate}
\end{frame}

\Subsection{Hexadecimal}

\Subsection{Fractions and Decimals}

\Subsection{Structured Data}

\begin{frame}{placeholder}
    strings
    audio
    images
    JSON
\end{frame}
    
