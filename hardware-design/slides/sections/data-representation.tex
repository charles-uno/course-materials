
\Section{Data Representation}

\begin{frame}{How do computers store information?}
    Computers work using electrical signals which can be on or off. If we let:
    \begin{itemize}
        \item On means 1
        \item Off means 0
    \end{itemize}
    We can use sequences of 0s and 1s to represent information
\end{frame}

\Subsection{Positive Integers in Binary}

\begin{frame}{Numbers in Base Ten (aka Decimal)}
    Let's start by talking about how we write numbers normally.

    For example, let's look at the number 109:
    \begin{align*}
        109 & = 100 \;+\; 0 \;+\; 9                                   \\
            & = 1 \Times 10^2 \;+\; 0 \Times 10^1 \;+\; 9 \Times 10^0
    \end{align*}
    This way of writing numbers is called base ten.
    We use ten digits (0 to 9 inclusive) and each position in the number is scaled by a power of ten.

    In terms of math, there is nothing special about base ten. We probably use it
    because we have ten fingers. Some ancient civilizations used sompletely
    different systems for counting!
\end{frame}

\begin{frame}{Numbers in Base Two (aka Binary)}
    Computers don't have fingers. They express everything as sequences of 1s and 0s. This format is called binary:

    \begin{align*}
        0b1101101 & =
        1 \Times 2^6 +
        1 \Times 2^5 +
        0 \Times 2^4 +
        1 \Times 2^3 +
        1 \Times 2^2 +
        0 \Times 2^1 +
        1 \Times 2^0                              \\
                  & = 64 + 32 + 0 + 8 + 4 + 0 + 1 \\
                  & = 109                         \\
    \end{align*}
    Importantly: we always use the prefix ``0b'' to avoid confusion when writing numbers in binary.

    1101 is one thousand one hundred and one

    0b1101 is thirteen
\end{frame}

\begin{frame}{Converting from Decimal to Binary}

    If the number is odd, append a 1 on the left. Otherwise, append a 0. Then
    divide your decimal number by two and ignore any remainder. Repeat until your
    decimal number is zero. \\

    For example, starting with 58:

    \begin{itemize}
        \item 58 is even, so append \Highlight{0}. Divide by two, leaving 29.
        \item 29 is odd, so append \Highlight{1}0. Divide by two, leaving 14.
        \item 14 is even, so append \Highlight{0}10. Divide by two, leaving 7.
        \item 7 is odd, so append \Highlight{1}010. Divide by two, leaving 3.
        \item 3 is odd, so append \Highlight{1}1010. Divide by two, leaving 1.
        \item 1 is odd, so append \Highlight{1}11010. Divide by two, leaving 0.
    \end{itemize}

    So 58 in binary is 0b111010.

\end{frame}

\begin{frame}{Converting from Binary to Decimal}

    We can confirm by converting back. The rightmost bit is worth 1, then 2, then
    4, and so on:

    \begin{align*}
        0b111010 & =
        1 \Times 2^5 +
        1 \Times 2^4 +
        1 \Times 2^3 +
        0 \Times 2^2 +
        1 \Times 2^1 +
        0 \Times 2^0                          \\
                 & = 32 + 16  + 8 + 0 + 2 + 0 \\
                 & = 58
    \end{align*}

\end{frame}

\begin{frame}{Addition in Binary}
    Binary addition works just like decimal addition.
    Start from the right, add straight down, and carry when you run out of digits.

    For example:
    \[
        \begin{array}{ccccccccc}
            1 & 1 & 1 & 1 & 1 &   &   &   &   \\
              & 0 & 0 & 1 & 1 & 1 & 0 & 0 & 0 \\
            + & 1 & 1 & 1 & 0 & 1 & 0 & 0 & 1 \\
            \hline
            1 & 0 & 0 & 1 & 0 & 0 & 0 & 0 & 1 \\
        \end{array}
    \]

    What happens if we try to perform this operation but we only have 8 bits to
    store the answer?
\end{frame}

\begin{frame}{Multiplication in Binary}
    Binary multiplication works just like decimal multiplication.
    Multiply the top number by the rightmost digit of the bottom number.
    Then move to the next line, add a zero, and repeat for the next digit.
    Finally, add up the lines. For example:
    \[
        \begin{array}{cccccc}
              &   &        & 1 & 1 & 0 \\
              &   & \times & 1 & 0 & 1 \\
            \hline
              &   &        & 1 & 1 & 0 \\
              &   & 0      & 0 & 0 & 0 \\
            + & 1 & 1      & 0 & 0 & 0 \\
            \hline
              & 1 & 1      & 1 & 1 & 0 \\
        \end{array}
    \]
\end{frame}

\begin{frame}{Division in Binary}
    To divide by 2, shift the bits one place to the right. \\

    To divide by 4, shift the bits two places to the right. \\

    That's about as deep as we go in this class. If you're curious to learn more,
    see the textbook.

\end{frame}

\begin{frame}{Exercises}
    Make sure to show your work.
    \vfill
    \begin{enumerate}
        \item Convert 0b1011 from binary to decimal. \vfill
        \item Convert 47 from decimal to binary. \vfill
        \item Add 0b1001 + 0b1011 in binary. Convert to decimal to check your work. \vfill
        \item Multiply 0b1101 $\times$ 0b110 in binary. Convert to decimal to verify your
              work. \vfill
    \end{enumerate}
\end{frame}

\Subsection{Negative Numbers}

\begin{frame}{First Attempt: Signed Magnitude}

    We can use the first digit to hold the sign, then the rest of the digits to
    hold magnitue:

    \begin{itemize}
        \item 0b\Highlight{0}1011001 is positive 10110001, so 89
        \item 0b\Highlight{1}1011001 is negative 1011001, so -89
    \end{itemize}

    This is nice and straightforward!

\end{frame}

\begin{frame}{Addition and Subtraction with Signed Magnitude}
    Adding positive numbers works just the same. But what happens if we throw a minus sign in there?

    For example, let's look at 12 - 5. First, rewrite it as 12 + (-5). Then:

    \[
        \begin{array}{ccccccccc}
              & 0 & 0 & 0 & 0 & 1 & 1 & 0 & 0 \\
            + & 1 & 0 & 0 & 0 & 0 & 1 & 0 & 1 \\
            \hline
              & ? & ? & ? & ? & ? & ? & ? & ? \\
        \end{array}
    \]

    We know the result should be 7 (0b00000111). But our regular rules for addition
    do not get us there.
\end{frame}

\begin{frame}{Zero with Signed Magnitude}

    Using signed magnitude, 0b00000000 is zero.

    And 0b10000000 is negative zero (which is also zero).

    Using this convention, we have to worry about the difference between numerical
    equality and bitwise equality. That seems pretty messy.
\end{frame}

\begin{frame}{Can We Do Better?}
    Signed magnitude was a swing and a miss. What do we want when we talk about negative numbers?
    \begin{itemize}
        \item We want positive numbers to work like we expect
        \item We want 0b00000000 to be zero, with no ambiguity
        \item We want addition and subtraction to work the same for positive and negative
    \end{itemize}
\end{frame}

\begin{frame}{Another Idea: Two's Complement}
    We know 1 - 1 = 0. Put another way, 1 + (-1) = 0. Can we work backwards from there to figure out how to write -1?
    \[
        \begin{array}{ccccccccc}
              & 0 & 0 & 0 & 0 & 0 & 0 & 0 & 1 \\
            + & 1 & 1 & 1 & 1 & 1 & 1 & 1 & 1 \\
            \hline
              & 0 & 0 & 0 & 0 & 0 & 0 & 0 & 0 \\
        \end{array}
    \]

    Remember: on a computer, our numbers have to fit in a set number of bits.
    Anything past that gets thrown away. This is called overflow. In this case,
    overflow works in our favor!
\end{frame}

\begin{frame}{Working Backwards from Zero}
    We can use the same trick to identify the rest of the negative numbers:
    \begin{itemize}
        \item 0b00000000 is 0
        \item 0b11111111 is -1
        \item 0b11111110 is -2
        \item 0b11111101 is -3
        \item 0b11111100 is -4
        \item etc
    \end{itemize}

\end{frame}

\begin{frame}{The Sign Bit (Again)}
    We only have 256 possible integers. How do we decide where the positives end and the negatives begin?

    Unsigned int: bits are worth 1, 2, 4, 8, 16, 32, 64, 128

    Signed magnitude: bits are worth $\pm1, \pm2, \pm4, \pm8, \pm16, \pm32, \pm64$,
    and the last bit tells us whether to use positive or negative (yikes)

    Two's complement: bits are worth 1, 2, 4, 8, 16, 32, 64, \Highlight{-128}

    Range of possible values is -128 to 127.
\end{frame}

\begin{frame}{Flipping Signs in Two's Complement}
    To flip the sign of a two's complement integer, flip all the bits then add 1. Ignore the overflow (if any). This works for both positive and negative numbers.

    For example:
    \begin{itemize}
        \item 55 is 0b01010111
        \item Flip the bits: 0b10101000, then add 1: 0b10101001
        \item So -55 is 0b10101001
        \item Flip the bits again: 0b01010110, add 1: 0b01010111
        \item So -(-55) = 55 = 0b01010111
    \end{itemize}

    What happens if we negate zero?
\end{frame}

\begin{frame}{Overflow}
    Overflow is important for two's complement. It ensures that positive zero and negative zero are the same. But it is also a constraint!

    Try adding 0b01101001 + 0b01011010:
    \[
        \begin{array}{ccccccccc}
              &   & 1 & 1 & 1 &   &   &   &   \\
              & 0 & 1 & 1 & 0 & 1 & 0 & 0 & 1 \\
            + & 0 & 1 & 0 & 1 & 1 & 0 & 1 & 0 \\
            \hline
              & 1 & 1 & 0 & 0 & 0 & 0 & 1 & 1 \\
        \end{array}
    \]

    These are two's complement numbers. Convert them to decimal. What just
    happened?

\end{frame}

\begin{frame}{Exercises}
    Make sure to show your work.
    \begin{enumerate}
        \item Convert 0b10010111 from two's complement binary to decimal.
        \item Convert -47 from decimal to two's complement binary.
        \item Subtract 0b01101001 - 0b10001100. These are two's complement binary numbers.
              Convert to decimal to check your result.
        \item Add 0b01101001 + 0b10001100. These are two's complement binary numbers. Convert
              to decimal to check your result.
    \end{enumerate}
\end{frame}

\Subsection{Hexadecimal}

\begin{frame}{What? Why?}

    Binary is how machines store data. But writing out binary by hand is a chore.
    In practice, it's often convenient to use hexadecimal (base 16) instead.

    \begin{itemize}
        \item Decimal uses ten digits, 0-9
        \item Binary uses two digits, 0 and 1
        \item Hexadecimal uses sixteen digits: 0-9 along with A-F
    \end{itemize}
    Hexadecimal values are always prefixed with ``0x'' to avoid ambiguity.
\end{frame}

\begin{frame}{Converting Between Binary and Hexadecimal}

    Converting back and forth between binary and hexadecimal does not require any
    math! Every four bits become one hex digit.
    \begin{align*}
        0b0000 & = 0x0 = 0 & 0b1000 & = 0x8 = 8  \\
        0b0001 & = 0x1 = 1 & 0b1001 & = 0x9 = 9  \\
        0b0010 & = 0x2 = 2 & 0b1010 & = 0xA = 10 \\
        0b0011 & = 0x3 = 3 & 0b1011 & = 0xB = 11 \\
        0b0100 & = 0x4 = 4 & 0b1100 & = 0xC = 12 \\
        0b0101 & = 0x5 = 5 & 0b1101 & = 0xD = 13 \\
        0b0110 & = 0x6 = 6 & 0b1110 & = 0xE = 14 \\
        0b0111 & = 0x7 = 7 & 0b1111 & = 0xF = 15 \\
    \end{align*}
\end{frame}

\begin{frame}{Converting Hexadecimal to Decimal}

    Hexadecimal is base 16. So each place corresponds to a power of 16.

    \begin{align*}
        0x3AB & = 3 \times 16^2 + 10 \times 16^1 + 11 \times 16^0 \\
              & = 3 \times 256 + 10 \times 16 + 11 \times 1       \\
              & = 768 + 160 + 11                                  \\
              & = 954                                             \\
    \end{align*}

    Hexadecimal is very compact! These numbers get big fast

\end{frame}

\begin{frame}{Arithmetic in Hexadecimal}

    Addition and multiplication work in hexadecimal just like they do in binary and
    decimal. Just be careful about carrying.

    \[
        \begin{array}{ccccc}
            1 &   &   & 1 &   \\
              & 4 & 1 & 7 & B \\
            + & C & 2 & 0 & F \\
            \hline
            1 & 0 & 3 & 8 & A \\
        \end{array}
    \]

\end{frame}

\begin{frame}{Exercises}
    Make sure to show your work.
    \begin{enumerate}
        \item TODO
    \end{enumerate}
\end{frame}

\Subsection{Fractions and Decimals}

\begin{frame}{The Decimal Point}

    Sometimes we need to store numbers that aren't integers. \\

    Before we worry about binary, what does 21.867 mean? \\

    \begin{align*}
        21.867 & = 2 \Times 10^1 + 1 \Times 10^0 + 8 \Times ?? + 6 \Times ?? + 7 \Times ??
    \end{align*}

\end{frame}

\begin{frame}{The Decimal Point}

    In decimal:
    \begin{align*}
        21.867 & = 2 \Times 10^1 + 1 \Times 10^0 + 8 \Times 10^{-1} + 6 \Times 10^{-2} + 7 \Times 10^{-3}             \\
               & =2 \Times 10 + 1 \Times 1 + 8 \Times \frac{1}{10} + 6 \Times \frac{1}{100} + 7 \Times \frac{1}{1000} \\
    \end{align*}

    In binary, we can use the same idea:
    \begin{align*}
        0b101.010 & = 1 \Times 2^2 + 0 \Times 2^1 + 1 \Times 2^0 + 0 \Times 2^{-1} + 1 \Times 2^{-2} + 0 \Times 2^{-3} \\
                  & = 1 \Times 4 + 0 + 1 \Times 1 + 0 + 1 \Times \frac{1}{4} + 0                                       \\
                  & = 5.25                                                                                             \\
    \end{align*}

\end{frame}

\begin{frame}{Fixed Point Representation}

    The computer just stores 1s and 0s, not decimal points. How do we distinguish
    these values?
    \begin{itemize}
        \item 0b0101110.1 = 46.5
        \item 0b010111.01 = 23.25
        \item 0b01011.101 = 11.625
    \end{itemize}

    One option is to specify the location of the decimal point ahead of time. For
    example, we could say that our eight-bit binary decimal always has two decimal
    bits. \\

    What is an upside of this approach? What is a downside?

\end{frame}

\begin{frame}{Floating Point Representation}

    Floating point numbers are a bit more complicated, but much more flexible. For
    a 16-bit floating point number we get:

    \begin{itemize}
        \item 1 bit for the sign
        \item 5 bits for the exponent (value -14 to 15)
        \item 10 bits for the significand (value 0 to 1023)
    \end{itemize}

    To turn those components into a value we use:
    \begin{align*}
        \text{value} & = (-1)^{\text{sign}} \times 2^{\text{exponent} - 15} \times \left(1 + \frac{\text{significand}}{1024} \right)
    \end{align*}

    Why offset the exponent?

    Why add 1 to the significand fraction?

\end{frame}

% Explanation here for the 127 offset: https://www.quora.com/Why-do-we-add-127-to-the-exponent-in-IEEE-754-floating-number-format-to-get-the-actual-exponent-value

\begin{frame}{Floating Point Examples}

    \begin{align*}
        \underbrace{0}_\text{sign} \; \underbrace{01111}_\text{exponent} \; \underbrace{0000000000}_\text{significand} & = (-1)^0 \times 2^{15 - 15} \times \left( 1 + \frac{0}{1024} \right) \\
                                                                                                                       & = 1 \times 2^0   \times \left( 1 + 0 \right)                         \\
                                                                                                                       & = 1                                                                  \\
    \end{align*}
    \begin{align*}
        0 \; 01101 \; 0101010101 & = (-1)^0 \times 2^{13 - 15} \times \left( 1 + \frac{341}{1024} \right) \\
                                 & \approx 0.33325195                                                     \\
    \end{align*}
    \begin{align*}
        1 \; 11110 \; 1111111111 & = (-1)^1 \times 2^{30 - 15} \times \left( 1 + \frac{1023}{1024} \right) \\
                                 & = -65504                                                                \\
    \end{align*}

    Why
\end{frame}

\begin{frame}{Floating Point Special Cases}

    \[
        \begin{array}{ccccr}
            0 & 00000 & 0000000000     & = 0          \\
            1 & 00000 & 0000000000     & = -0         \\
            0 & 11111 & 0000000000     & = \infty     \\
            1 & 11111 & 0000000000     & = -\infty    \\
            0 & 11111 & \text{nonzero} & = \text{NaN} \\
        \end{array}
    \]

    There's more to it than this. If you're curious, read up on subnormal numbers
    \href{https://en.wikipedia.org/wiki/Half-precision_floating-point_format}{here}

\end{frame}

\begin{frame}{Floating Point for Big Integers}

    If we're dealing with big numbers, floating point often works better than fixed
    point. This is true even if both formats use the same number of bits.

    \begin{itemize}
        \item 8-bit unsigned integer: max value 255
        \item 16-bit unsigned integer: max value 65,535
        \item 32-bit unsigned integer: max value 4.2 billion % 4,294,967,295
        \item 16-bit floating point: max value 65,504 (also handles negatives, fractions)
        \item 32-bit floating point: max value 340 billion billion billion billion (also handles negatives, fractions)
          % 340,282,346,638,528,859,811,704,183,484,516,925,440 
    \end{itemize}

\end{frame}

\begin{frame}{Floating Point Limitations}

    So what's the catch?

\end{frame}

\begin{frame}{Exercises}
    Make sure to show your work.
    \begin{enumerate}
        \item Convert 0b100110.10 from fixed-point binary to decimal.
        \item Convert 23.75 from decimal to fixed-point binary.
        \item Add fixed-point binary numbers 0b101101.01 + 0b010011.11. Convert to decimal to
              confirm your result.
        \item Show that the 16-bit floating point expression ${1 \; 01000 \; 1100000000}$ is
              equal to -3.5 in decimal.
    \end{enumerate}
\end{frame}

\Subsection{Structured Data}

\begin{frame}{placeholder}
    strings
    audio
    images
    JSON
\end{frame}

