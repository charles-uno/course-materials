
\Section{Assembly Fundamentals}

\begin{frame}{Hello World in C}
\end{frame}


\begin{frame}[fragile]{Compiling C Code}

The following command compiles a program in C:

\begin{verbatim}
    gcc hello.c -o hello
\end{verbatim}

This creates several intermediate files, eventually creating an executable file (which we can run).
`hello.c'
preprocessor
`hello.i'
compiler
`hello.s'
assembler
`hello.o'
linker
`hello'

\end{frame}

\begin{frame}{What is Assembly?}
\end{frame}


\begin{frame}{Why Learn Assembly?}
\end{frame}


\begin{frame}[fragile]{Hello World in Assembly}
    \begin{alltt}
\Highlight{@ global read-only data (aka constants)}
.section .rodata
greeting: .ascii "Hello World!\n\0"

\Highlight{@ execution starts here}
.section .text
.global main
main:
    \Highlight{@ load the string address to r0}
    ldr r0, =greeting
    \Highlight{@ print the string from r0}
    bl printf
    \Highlight{@ return 0 (normal exit status)}
    mov r0, \#0
    bl exit
\end{alltt}
\end{frame}

\begin{frame}{Hello World with Instruction Addresses}
    \begin{alltt}    
        \begin{tabular}{ r | l  }
            - & \Highlight{@ global read-only data (aka constants)} \\
            - & .section .rodata \\
            - & greeting: .ascii "Hello World!\n\0" \\
            - & \Highlight{@ execution starts here} \\
            - & .section .text \\
            - & .global main \\
            - & main: \\
            - & \quad \Highlight{@ load the string address to r0} \\
            0x5000 & \quad ldr r0, =greeting \\
            - & \quad \Highlight{@ print the string from r0} \\
            0x5004 & \quad bl printf \\
            - & \quad \Highlight{@ return 0 (normal exit status)} \\
            0x5008 & \quad mov r0, \#0 \\
            0x500c & \quad bl exit \\
                \end{tabular}
        \end{alltt}
    \end{frame}
    

\begin{frame}{Memory Diagram}
\begin{alltt}
    \begin{tabular}{ r | l p{5mm} r | l }
        \multicolumn{2}{c}{Registers} && \multicolumn{2}{c}{Memory} \\
        r0 & fizz && 0x3fe8 & buzz \\
        r1 & fizz && 0x3fec & buzz \\
        r2 & fizz && 0x3ff0 & buzz \\
        r3 & fizz && 0x3ff4 & buzz \\
        sp & fizz && 0x3ff8 & buzz \\
        pc & fizz && 0x3ffc & buzz \\
        lr & fizz && 0x4000 & buzz \\
        \end{tabular}
    \end{alltt}
\end{frame}

\begin{frame}{Hello World Walkthrough (0)}
    \begin{alltt}
        \begin{tabular}{ r | l p{2mm} r | l p{2mm} r | l }
            \multicolumn{2}{c}{Registers} && \multicolumn{2}{c}{Memory} && \multicolumn{2}{c}{Memory} \\
            r0 & ? && 0x3fe8 & ? && 0x4fe8 & ldr r0, =greeting \\
            r1 & ? && 0x3fec & ? && 0x4fec & bl printf \\
            r2 & ? && 0x3ff0 & Hell && 0x4ff0 & mov r0, \#0 \\
            r3 & ? && 0x3ff4 & o Wo && 0x4ff4 & bl exit \\
            sp & ? && 0x3ff8 & rld! && 0x4ff8 & ? \\
            pc & 0x4fe8 && 0x3ffc & {\textbackslash}0 && 0x4ffc & ? \\
            lr & ? && 0x4000 & ? && 0x5000 & ? \\
            \end{tabular}
        \end{alltt}
\end{frame}



\begin{frame}{Hello World Walkthrough (1)}
    \begin{alltt}
        \begin{tabular}{ r | l p{2mm} r | l p{2mm} r | l }
            \multicolumn{2}{c}{Registers} && \multicolumn{2}{c}{Memory} && \multicolumn{2}{c}{Memory} \\
            r0 & 0x3ff0 && 0x3fe8 & ? && 0x4fe8 & ldr r0, =greeting \\
            r1 & ? && 0x3fec & ? && 0x4fec & bl printf \\
            r2 & ? && 0x3ff0 & Hell && 0x4ff0 & mov r0, \#0 \\
            r3 & ? && 0x3ff4 & o Wo && 0x4ff4 & bl exit \\
            sp & ? && 0x3ff8 & rld! && 0x4ff8 & ? \\
            pc & 0x4fec && 0x3ffc & {\textbackslash}0 && 0x4ffc & ? \\
            lr & ? && 0x4000 & ? && 0x5000 & ? \\
            \end{tabular}
        \end{alltt}

        Execute \texttt{ldr} to load the address of global constant \texttt{greeting} to \texttt{r0}
        
        Also increment \texttt{pc}

\end{frame}

\begin{frame}{Hello World Walkthrough (2)}
    \begin{alltt}
        \begin{tabular}{ r | l p{2mm} r | l p{2mm} r | l }
            \multicolumn{2}{c}{Registers} && \multicolumn{2}{c}{Memory} && \multicolumn{2}{c}{Memory} \\
            r0 & ? && 0x3fe8 & ? && 0x4fe8 & ldr r0, =greeting \\
            r1 & ? && 0x3fec & ? && 0x4fec & bl printf \\
            r2 & ? && 0x3ff0 & Hell && 0x4ff0 & mov r0, \#0 \\
            r3 & ? && 0x3ff4 & o Wo && 0x4ff4 & bl exit \\
            sp & ? && 0x3ff8 & rld! && 0x4ff8 & ? \\
            pc & 0x4ff0 && 0x3ffc & {\textbackslash}0 && 0x4ffc & ? \\
            lr & ? && 0x4000 & ? && 0x5000 & ? \\
            \end{tabular}
        \end{alltt}

        Use \texttt{bl} to call the built-in function \texttt{printf}, which looks at the address from \texttt{r0} and prints the data as a null-terminated string
        
        Also increment \texttt{pc}


\end{frame}

\begin{frame}{Hello World Walkthrough (3)}
    \begin{alltt}
        \begin{tabular}{ r | l p{2mm} r | l p{2mm} r | l }
            \multicolumn{2}{c}{Registers} && \multicolumn{2}{c}{Memory} && \multicolumn{2}{c}{Memory} \\
            r0 & 0 && 0x3fe8 & ? && 0x4fe8 & ldr r0, =greeting \\
            r1 & ? && 0x3fec & ? && 0x4fec & bl printf \\
            r2 & ? && 0x3ff0 & Hell && 0x4ff0 & mov r0, \#0 \\
            r3 & ? && 0x3ff4 & o Wo && 0x4ff4 & bl exit \\
            sp & ? && 0x3ff8 & rld! && 0x4ff8 & ? \\
            pc & 0x4ff4 && 0x3ffc & {\textbackslash}0 && 0x4ffc & ? \\
            lr & ? && 0x4000 & ? && 0x5000 & ? \\
            \end{tabular}
        \end{alltt}

        Move the value zero to \texttt{r0}

        Also increment \texttt{pc}


\end{frame}

\begin{frame}{Hello World Walkthrough (4)}
    \begin{alltt}
        \begin{tabular}{ r | l p{2mm} r | l p{2mm} r | l }
            \multicolumn{2}{c}{Registers} && \multicolumn{2}{c}{Memory} && \multicolumn{2}{c}{Memory} \\
            r0 & 0 && 0x3fe8 & ? && 0x4fe8 & ldr r0, =greeting \\
            r1 & ? && 0x3fec & ? && 0x4fec & bl printf \\
            r2 & ? && 0x3ff0 & Hell && 0x4ff0 & mov r0, \#0 \\
            r3 & ? && 0x3ff4 & o Wo && 0x4ff4 & bl exit \\
            sp & ? && 0x3ff8 & rld! && 0x4ff8 & ? \\
            pc & 0x4fe8 && 0x3ffc & {\textbackslash}0 && 0x4ffc & ? \\
            lr & ? && 0x4000 & ? && 0x5000 & ? \\
            \end{tabular}
        \end{alltt}

        Exit from the program. The value in \texttt{r0} is the returncode

        Update \texttt{pc} to point to instructions for whatever called our program

\end{frame}



\Subsection{Global Variables}


\Subsection{Printf and Scanf}


\Subsection{Memory Diagrams}



\Subsection{Automated Testing}






